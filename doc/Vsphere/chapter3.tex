
%
\section{インストール}
\label{sec:install}
V-Sphereのインストールは,MPI通信ライブラリのインストールが終了したあとに行う.
V-Sphereは,単精度版と倍精度版を別々に用意する必要がある.

\begin{enumerate}

\item configure\footnote{ここでは,MacOSX,IA-64モード,コンパイラの環境として,/opt/intel/Compiler/11.1/089のディレクトリを仮定,コンパイルオプション\index{コンパイルオプション}として,-O3 を指定.icc, icpc, ifortへのパスは既に指定していると仮定する.特別な仕様のマシン向けのインストールには,makefile.specを使う.これは,特別な仕様のマシン向け(p4-dev)である.\hyperlink{tgt:BGL}{BlueGene/L}の項やV-Sphereのマニュアルを参照のこと.アーキテクチャに相応しいコンパイルオプションがわからない場合は,「CXXFLAGS=-O3」「F90FLAGS=-O3」だけでも可.Intel Compiler C/C++ version 10は,multibyteの文字コードの対応が不充分のため,C/C++のコンパイルオプションで-no-multibyte-charsを明示的に指示する.Mac用のIntel Compiler 11.0では不要だが,Linuxでは必要.
}

\paragraph{コマンドライン}
コマンドラインで作業する場合には,次のようにタイプする.インストールディレクトリには\verb|/usr/local/sphere/|を指定している.もし,\verb|/usr/local/|領域へのアクセス権限がない場合には,各ユーザが書き込めるところを指定する.また,LDFLAGSには適切なパスを指定する.

{\small
\begin{program}
$ ./configure --prefix=/usr/local/sphere \
              --with-comp=INTEL \
              --with-ompi=/opt/openmpi  \
              CC=icc \
              CFLAGS='-O3'\
              CXX=icpc \
              CXXFLAGS='-O3'\
              FC=ifort \
              FCFLAGS='-O3'\
              F90=ifort \
              F90FLAGS='-O3'\
              LDFLAGS=-L/opt/intel/Compiler/11.1/089/lib
\end{program}
}

configureでインストール先を指定すると,指定したインストールディレクトリはソースディレクトリの\verb|sphconfig/sph-cfg.xml|に記録される.このファイルは,インストールディレクトリの\verb|config/sph-cfg.xml|にコピーされ,sphPrjToolのresetコマンドで参照される.このため,一旦インストールディレクトリを指定したら,移動したりリネームすると,resetコマンドが正しく機能しなくなるので注意する.

%
\paragraph{倍精度モジュール}
倍精度計算をする場合には,V-Sphereは倍精度モジュールとしてコンパイルする必要がある.configure時のオプションに\verb|--with-real=double|を追加する.このオプションにより,C/C++コンパイラに\verb|-DREAL_IS_DOUBLE|,Fortranコンパイラには\verb|-r8|がコンパイルオプションとして自動的に追加される.

\paragraph{シェル}
次のインストールシェルは,引数としてインストールディレクトリを指定する.

{\small
\begin{program}
$ configure.sh /usr/local/sphere
\end{program}
}

{\small
\begin{program}
------------------------------------
configure.sh (mpich + float)
------------------------------------
#! /bin/sh
./configure --prefix=$1 \
            --with-comp=INTEL \
            --with-mpich=/usr/local/mpich \
            CC=icc \
            CFLAGS=-O3\
            CXX=icpc \
            CXXFLAGS=-O3\
            FC=ifort \
            FCFLAGS=-O3\
            F90=ifort \
            F90FLAGS=-O3\
            LDFLAGS=-L/opt/intel/Compiler/11.0/059/lib

------------------------------------
configure.sh (mpich + double)
------------------------------------
#! /bin/sh
./configure --prefix=$1 \
            --with-comp=INTEL \
            --with-real=double \
            --with-mpich=/usr/local/mpich \
            CC=icc \
            CFLAGS=-O3\
            CXX=icpc \
            CXXFLAGS=-O3\
            FC=ifort \
            FCFLAGS=-O3\
            F90=ifort \
            F90FLAGS=-O3\
            LDFLAGS=-L/opt/intel/Compiler/11.0/059/lib

------------------------------------
configure.sh (OpenMPI + float)
------------------------------------
#! /bin/sh
./configure --prefix=$1 \
            --with-comp=INTEL \
            --with-ompi=/usr/local/ompi \
            CC=icc \
            CFLAGS=-O3\
            CXX=icpc \
            CXXFLAGS=-O3\
            FC=ifort \
            FCFLAGS=-O3\
            F90=ifort \
            F90FLAGS=-O3\
            LDFLAGS=-L/opt/intel/Compiler/11.0/059/lib

------------------------------------
configure.sh (OpenMPI + double)
------------------------------------
#! /bin/sh
./configure --prefix=$1 \
            --with-comp=INTEL \
            --with-real=double \
            --with-ompi=/usr/local/ompi \
            CC=icc \
            CFLAGS=-O3\
            CXX=icpc \
            CXXFLAGS=-O3\
            FC=ifort \
            FCFLAGS=-O3\
            F90=ifort \
            F90FLAGS=-O3\
            LDFLAGS=-L/opt/intel/Compiler/11.0/059/lib
\end{program}
}


\item make
\footnote{make時にlibimf.soが見つからないなどのメッセージが出る場合は,ユーザのLD\_LIBRARY\_PATHにパス を加えておく.\\
"LD\_LIBRARY\_PATH=/opt/intel/Compiler/11.0/056/lib:/usr/local/mpich/lib"}

{\small
\begin{program}
$ make
\end{program}
}
\vspace{\baselineskip}

\item Install
{\small
\begin{program}
$ sudo make install または make install
\end{program}
}

V-Sphereライブラリがインストールディレクトリにインストールされると,次に示すようなコンパイルに関する情報がソースディレクトリの\verb|sphconfig/sph-cfg.xml|に記述される.
{\small
\begin{program}
<SphereEnvironment>
  <Param name="SPHEREDIR" dtype="STRING" value= "/usr/local/sphere"/>
  <Param name="CXX" dtype="STRING" value="icpc"/>
  <Param name="CXXFLAGS" dtype="STRING" value="-O3"/>
  <Param name="CC" dtype="STRING" value="icc"/>
  <Param name="CFLAGS" dtype="STRING" value="-O3"/>
  <Param name="FC" dtype="STRING" value="ifort"/>
  <Param name="FCFLAGS" dtype="STRING" value="-O3"/>
  <Param name="F90" dtype="STRING" value="ifort"/>
  <Param name="F90FLAGS" dtype="STRING" value="-O3"/>
  <Param name="LDFLAGS" dtype="STRING" value="-L/opt/intel/Compiler/11.1/067/lib"/>
  <Param name="SPH_DEVICE" dtype="STRING" value="Snow_Leopard"/>
  <Param name="MPICH_DIR" dtype="STRING" value="/usr/local/ompi"/>
  <Param name="MPICH_CFLAGS" dtype="STRING" value="-I/usr/local/ompi/include"/>
  <Param name="MPICH_LDFLAGS" dtype="STRING" value="-L/usr/local/ompi/lib"/>
  <Param name="MPICH_LIBS" dtype="STRING" value="-lmpi"/>
  <Param name="XML2FLAGS" dtype="STRING" value="-I/usr/include/libxml2"/>
  <Param name="XML2LIBS" dtype="STRING" value="-lxml2 -lz -lpthread -licucore -lm"/>
  <Param name="SPHERE_CFLAGS" dtype="STRING" value="-DSKL_TIME_MEASURED -D_CATCH_BAD_ALLOC 
               -I/usr/local/vsph175/include"/>
  <Param name="SPHERE_LDFLAGS" dtype="STRING" value="-L/usr/local/vsph175/lib"/>
  <Param name="SPHERE_LIBS" dtype="STRING" value="-lsphapp -lsphbase -lsphls -lsphfio -lsphdc 
               -lsphcrd -lsphcfg -lsphftt -lsphvcar"/>
  <Param name="LIBS" dtype="STRING" value="-lifport -lifcore"/>
</SphereEnvironment>
\end{program} 
}

%
\paragraph{マニュアルインストール}
別の方法として,Config.specを編集する.
ポイントは,libxml2とinstallコマンドのパス.
{\small
\begin{program}
$ xml2-config --libs
\end{program}
}
を実行してメッセージが返ればOK.
{\small
\begin{program}
$ which install
\end{program}
}
インストールコマンドがあればOK.
その後,
{\small
\begin{program}
$ make -f Makefile.spec
# make -f Makefile.spec install
\end{program}
}
\vspace{\baselineskip}

\item インストールに失敗する場合\\
makeに失敗して,何度もインストールしているとMakeで使用する環境変数がおかしくなることがある.
やり直すときは、コンフィギュレーションをクリアし,最初からインストール作業を行う.

{\small
\begin{program}
$ make distclean  (コンフィギュレーションのクリア)
\end{program}
}

ただし,再インストールの場合はtarボールから解凍して再試行する方がより安全.
また,上記の\verb|make distclean|を実行すると,設定ファイルが全て消去される.

\end{enumerate}

%%
\section{アンインストール}
V-Sphereをアンインストール\index{アンインストール!V-Sphereの@V-Sphereの---}する場合には,インストールしたディレクトリ(configureでオプション指定したディレクトリ)のsphereを削除する.
