
%
\section{mpich1}
\label{sec:mpich1}

\begin{enumerate}

\item 簡単な手順\footnote{Macの場合には,-rsh=ssh を使用する.Intel Compiler C/C++ 10.1では,multibyteの文字コードの対応が不充分のようで、デフォルトの設定ではうまく動かない.
このため,configure時には次のオプションをつける.version 11.0, rev.056以降では,-no-multibyte-charsは不要.\\
"-cflags='-O3 -no-multibyte-chars' -c++flags='-O3 -no-multibyte-chars'"
}

\begin{indentation}{3zw}{0zw}
Intel Mac, Intel Compiler 11.1の場合について示す.

{\small
\begin{program}
$ ./configure --prefix=/usr/local/mpich -cc=icc -c++=icpc -fc=ifort -f90=ifort \
              -cflags=-O3 -c++flags=-O3 -rsh=ssh
$ make
$ sudo make install
\end{program}
}
\end{indentation}
\vspace{3mm}

\item トラブルシューティング\index{トラブルシューティング}\\
\begin{indentation}{3zw}{0zw}

\begin{itemize}
\item gcc でもiccでも可能.
configure時に環境変数を使う指示があるが,このconfigure はコマンドラインで指定すること.
\footnote{Intel Compiler C/C++ ver.10までは,32ビット用コンパイラと64ビット用コンパイラが混在している.Intel64のプラットホームで,cc(32bit), cce(64bit) の両方がある場合,先にcceを記述する.\\
"PATH=/usr/local/mpich/bin:/opt/intel/cce/10.1/bin:/opt/intel/cc/10.1/bin"
}

{\small
\begin{program}
--prefix=/usr/local/mpich
-fc=ifort
-f90=ifort
-rsh=rsh or ssh
\end{program}
}
\vspace{3mm}

\item fortranの設定ができていない場合には, /usr/local/mpich/bin配下にmpif90などのインクルードファイル,include配下にmpif.h, f90choice/などができていないので,この点を確認する.
configureの内容は,bin/mpireconfig.datの先頭付近にコメントとして書かれている.両方を試す場合には,
{\small
\begin{program}
/usr/local/mpich.gcc
/usr/local/mpich.intel
\end{program}
}
などをつくり,/usr/local/mpich にスタティックリンクを張る.
{\small
\begin{program}
$ cd /usr/local
$ sudo ln -s mpich.intel/ mpich
\end{program}
}
\vspace{3mm}

\item システムの設定で,場合によっては/usr/bin/mpirun\index{mpirun}がコールされる場合がある.
これはPATHで/usr/local/mpich/binを先に見るように設定しておく.
\vspace{3mm}

\item sphereのコンパイルで,libxml2のpthread関連のエラーが出ることがある.
make.IA64\_linuxなどで,XMLLIBSの最後に-lpthreadを追加して対応する.
\vspace{3mm}

\item リンクエラーが出た場合,
{\small
\begin{program}
$ nm *.a | less
\end{program}
}
などで,該当の関数を探す.
記号Tを全て探して,対象のアーカイブをリンクするようにmake.*を変更する.
具体的には,FCLIBS=...
\vspace{3mm}

\item LinuxでIntel Compiler 10.x/11.xの場合,
{\small
\begin{program}
-cflags='-O3 -no-multibyte-chars' -c++flags='-O3 -no-multibyte-chars'
\end{program}
}
\vspace{3mm}

\item Intel mac OSX 10.5で,Intel Compiler 11.0/11.1の場合,rshは必ずsshのこと.また,Intel Compiler 10.1の場合には,Linuxの場合と同様に"-no-multibyte-chars"が必要.
{\small
\begin{program}
$ ./configure --prefix=/usr/local/mpich -cc=icc -c++=icpc -fc=ifort -f90=ifort \
              -cflags=-O3 -c++flags=-O3 -rsh=ssh
\end{program}
}
\vspace{3mm}

\end{itemize}
\end{indentation}

\end{enumerate}

%%%%%%%
\begin{comment}
\section{mpich2}
\label{sec:mpich2}

\begin{enumerate}

\item コンパイル環境の設定\footnote{Windows IA32\index{Windows}においてmpich2をインストールするためには,
「Microsoft Visual Studio 2008」\index{Visual Studio}\\
あるいは,\\
「Microsoft Visual C++ (2005または)2008 SPI 再頒布可能パッケージ(x86)」\\
「Windows Installer 3.1 Redistributable(v2) - 日本語」\\
「Microsoft .NET Framework 3.5」\\
の3点を前もってインストールすることが必要.詳細は,V-SphereのWindowsマニュアルを参照のこと.}

\begin{itemize}
\item /usr/local/mpich2にインストールし,/usr/local/mpichにリンクする
\item ”ssm”(sockets and shared memory) クラスタ間のプロセス間通信はソケット通信,SMP上のプロセス間通信は共有メモリを利用するデバイスとチャネルを選択する.\footnote{mpich2では,特定のチャネルと対になったデバイスにより通信が行われる.デフォルトでは,ch3デバイス上のsockチャネルが設定される.ch3デバイス上のチャネルは他にも,nemesis, shm(shared memory), ssm(sockets and shared memory) がある.ssmでは,マルチスレッドは現時点でのサポートなし.MPI\_THREAD\_MULTIPLEは,sockとnemesisのみ.現リリース(mpich2-1.0.7-rc1)では,デフォルトチャネル(sock)以外を利用する場合,環境変数にチャネル名をセットすること.\\
\verb|$ export MPICH_CH3CHANNEL=SSM|
}
\item 共有ライブラリを利用(コンパイラはgccのみに対応)
\item 環境変数のセット(インストール時にテンポラリに)\footnote{共有メモリオプションを利用する場合には,gccのサポートしかないので,gcc を利用する.}
\end{itemize}
\vspace{\baselineskip}

\item 簡単な手順
\begin{indentation}{3zw}{0zw}
{\small
\begin{verbatim}
$ export CC=gcc4
$ export CFLAGS=’-O3 -m64’
$ export CXX=g++
$ export CXXFLAGS=’-O3 -m64’
$ export F77=ifort
$ export FFLAGS=’-O3’
$ export F90=ifort
$ export F90FLAGS=’-O3’
$ export MPICH_CH3CHANNEL=SSM
$ export MPICH_DEVICE=’--with-device=ch3:ssm’

$ ./configure --prefix=/usr/local/mpich2
              --enable-error-checking=no
              --enable-error-messages=none
              --enable-timing=none
              --enable-g=none
              --enable-fast
              --enable-ndebug
              --enable-f77
              --enable-f90
              --enable-cxx
              --enable-romio
              --enable-threads=single
              --with-thread-package=posix
\end{verbatim}
\verb|              --enable-sharedlibs=gcc|\footnote{Mac OSXの場合は\quad \verb|--enable-sharedlibs=osx-gcc|
}

\begin{verbatim}
$ make
$ sudo make install

$ cd /usr/local
$ sudo ln -s mpich2 mpich
\end{verbatim}
}
\end{indentation}
\vspace{\baselineskip}

\item トラブルシューティング
{\small
\begin{itemize}
\item Windows\index{Windows}でSP3をインストールしてある場合\\
Windows Installer 3.1 Redistibutable(v2)-日本語をインストールする場合に,「このシステムのService Packが適用しようとしている更新より新しいバージョンであることが検出されました。この更新をインストールする必要はありません」とメッセージがでて,インストールされていないかもしれないがインストール自体は完了している.
\vspace{\baselineskip}
\item gccのエラー\\
\verb|$ export CC=gcc4|\quad では、C compilerが見つからないというエラーが出る場合がある.
\begin{verbatim}
checking for gcc... gcc4
checking for C compiler default output file name...
configure: error: C compiler cannot create executables
\end{verbatim}
この場合,gccを用いる.
\vspace{\baselineskip}
\end{itemize}

}
\end{enumerate}
\end{comment}
%%%%%%%

\section{mpich2}
\label{sec:mpich2}
mpich2を例に説明する.

%
\begin{enumerate}
\item 簡単な手順\\
MPICH2のWEBサイト\footnote{\url{http://www.mcs.anl.gov/research/projects/mpich2/}\\
2011年5月12日現在,安定リリース版はmpich2-1.3.2p1.tar.gzである.}から,ソースをダウンロードして解凍する.

作成されたディレクトリに入り,configureのために,次のようなスクリプトを用意し,実行する.
インストールディレクトリは,/usr/local/mpich2とする.

%
\begin{indentation}{3zw}{0zw}
{\small
\begin{program}
$ cat config_mpich2.sh
-----------------------------------
#!/bin/sh
export CC=icc
export CFLAGS=-O3
export CXX=icpc
export CXXFLAGS=-O3
export F77=ifort
export FFLAGS=-O3
export FC=ifort
export FCFLAGS=-O3
#
./configure --prefix=$1
-----------------------------------

$ ./config_mpich2.sh /usr/local/mpich2

$ make

$ sudo make install
\end{program}
}
\end{indentation}

\end{enumerate}


%%%
\section{OpenMPI}
\label{sec:ompi}
OpenMPI-1.3.2を例に説明する.

%
\begin{enumerate}
\item 簡単な手順\\
configureのために,次のようなスクリプトを用意し,実行する.インストールディレクトリは/usr/local/ompiとする.

\begin{indentation}{3zw}{0zw}
{\small
\begin{program}
$ cat config_ompi.sh
------------------------------
#!/bin/sh
export CC=icc
export CFLAGS=-O3
export CXX=icpc
export CXXFLAGS=-O3
export F77=ifort
export FFLAGS=-O3
export FC=ifort
export FCFLAGS=-O3
#
./configure --prefix=$1 
------------------------------

$ ./config_ompi.sh /usr/local/ompi

$ make

$ sudo make install
\end{program}
}
\end{indentation}
\vspace{\baselineskip}

\item PATHの設定\\
実行時のmpiexec\footnote{mpirunでも動く.}が正しいパスになっているかどうかをwhichコマンドで確認する.
\begin{indentation}{3zw}{0zw}
{\small
\begin{program}
$ which mpiexec
/usr/bin/mpiexec
\end{program}
}
\end{indentation}

Mac OSXの場合には上記のように,デフォルトでインストールされているOpenMPIの方を見に行くので,インストールしたOpenMPIのPATHを最初の方に書いておく.
\begin{indentation}{3zw}{0zw}
{\small
\begin{program}
$ cat .bash_priofile

#! /bin/sh
PATH=~/bin:~/bin/script:/usr/local/ompi/bin:${PATH}; export PATH
...
. ~/.bashrc

\end{program}
}
\end{indentation}

\end{enumerate}

