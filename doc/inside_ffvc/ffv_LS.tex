\graphicspath{{./fig_EBCS/}}
%
\section{圧力のPoisson方程式}

圧力のPoisson方程式は \textbf{式(\ref{eq:Poisson_from_FS})}を再掲すると,

\begin{equation}
\nabla \left( \nabla p^{\,n+1} \right) \,=\, \frac{1}{\Delta t} div\, (u^{\,*}) \equiv \psi
\label{eq:Poisson_from_FS2}
\end{equation}

ノイマン条件とディリクレ条件のフラグを導入し,\textbf{式(\ref{eq:ebcs_poisson3})}のように書ける.
これを対角項が1となるようにスケーリングして,連立一次方程式$A\underline{x} = \underline{b}$を得る.

\begin{equation}
\underbrace{ p  -  \frac{1}{\sum \limits_l \phi^{\,N}_{\,l}} \sum \limits_l {\left( p \,\phi^{\,D} \,\phi^{\,N} \right)}_{\,l} } \limits_{A \underline{x} }
\,=\,
 \displaystyle { 
\underbrace{ 
\frac{1}{\sum \limits_l \phi^{\,N}_{\,l}} 
\{ 
- h^2 \,\psi \,-\,  \psi^{BC}
\}
} \limits_{ \underline{b} }
}
\label{eq:ebcs_poisson_4}
\end{equation}


反復回数を$k$としてSOR法系の反復式を構成すると,

\begin{equation}
\left.
\begin{array}{lll}
\vspace{2mm}
\tilde{p} & = & \displaystyle {  \underline{b} \, + \,
 \frac{1}{\sum \limits_l \left( \, \phi^{\,N}_{\,l} \, \right)} \sum \limits_l {\left( p \,\phi^{\,D} \,\phi^{\,N}\right)}_{\,l} 
 \,=\,
\frac{1}{\sum \limits_l \left( \, \phi^{\,N}_{\,l} \, \right)} 
\left \{ \,
\sum \limits_l {\left( p \,\phi^{\,D} \,\phi^{\,N}\right)}_{\,l} 
- h^2 \,\psi \,-\,  \psi^{BC}
\, \right \} } \\
\vspace{2mm}
\Delta p & = & \tilde{p} \,-\, p^{\,k} \\
\vspace{5mm}
p^{\,k+1} & = & p^{\,k} \,+\, \omega\, \,\Delta p \\

\end{array} \qquad \right \}
\label{eq:poisson_iteration}
\end{equation}


%
\section{GMRES}

GMRES(Generalized Minimum Residual)は,大型疎行列を係数行列にもつ連立一次方程式$A \underline{x} = \underline{b}$の解法アルゴリズムで,Krylov部分空間法に分類される.
係数行列$A$は正則な$m \times m$行列であり,非対称行列($A \neq A^T$)でもよい.右辺の定数項は$\|\underline{b}\|_2=1$となるよう正規化する.

n次のKrylov部分空間は
\begin{equation}
K_n \,=\, span \{ b,\, Ab,\, A^2b,\, ...,\, A^{n-1}b \}
\label{eq:subspace}
\end{equation}

\noindent である.GMRES法は,各反復毎に残差ノルムを最小にするように解ベクトル$\underline{x}_n$を選ぶので,残差ノルムは単調に減少する特徴をもつ.
このとき,Arnoldi過程により$K_n$の基底を構成する正規直交ベクトル列$\underline{Q}_n\,=\,\{\underline{q}_1,\,\underline{q}_2,\,...,\,\underline{q}_n\}$を計算する.
$\underline{Q}_n$は$m\times n$行列である.
また,標準基底$\mathbb{R}^{n+1}$の第一ベクトルを次のように定める.

\begin{equation}
\underline{e}_1 \,=\, (1,\,0,\,0,\,...,\,0)
\label{eq:1st basis}
\end{equation}

Arnoldi過程では,次の関係を満たすHessenberg行列$\tilde{H}_n$が作られる.

\begin{equation}
A Q_n \,=\, Q_{n+1} \tilde{H}_n
\label{eq:Hessenberg matrix}
\end{equation}

$\underline{x}_n\,=\,Q_n \underline{y}_n$なので,次式が成り立つ.

\begin{equation}
\| A \underline{x}_n - \underline{b} \| \,=\, \| \tilde{H}_n \underline{y}_n - \underline{e}_1 \|
\label{eq:Hessenberg norm}
\end{equation}

したがって,近似解$\underline{x}_n$は残差$\underline{r}_n$を最小化する(最小自乗法など)ことにより計算できる.

\begin{equation}
\underline{r}_n \,=\, \tilde{H}_n \underline{y}_n - \underline{e}_1 
\label{eq:Hessenberg redsidual}
\end{equation}

Krylov部分空間法は$A^{-1}$を反転して$\underline{x}$を直接求めるのではなく,前処理行列$C$を用いた方程式$A C^{-1} (C \underline{x}) = \underline{b}$の近似解を求める.
$C$の逆行列が簡単に推定でき,前処理行列を係数行列とする連立一次方程式$C \underline{y} = \underline{z}$がSOR法などの数回の緩和で近似解が得られると仮定する.つまり,任意のベクトル$\underline{z}$に対して$C^{-1}\underline{z}$が比較的簡単に計算できるとしよう.












