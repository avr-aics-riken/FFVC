\section{Fortran90ユーザープログラムでの利用方法}
\label{use_f90}
以下に,CPMライブラリのFortran90インターフェイスの説明を示します.

なお,以降のFortran90インターフェイスの説明では,各メソッドをFortran90サブルーチン
の形式で記述していますが,実際には「{\tt extern "C"}」定義のC言語型インターフェイスの
C++メソッドとして記述されています.インターフェイスメソッドでは,cpm\_ParaManager
クラスの唯一のインスタンスを経由して,CPMライブラリの各種機能にアクセスしています.


\subsection{実数型}
\label{realtypeF}
CPMライブラリのFortran90インターフェイスで扱われる実数型は,全てREAL\_TYPE型
(\ref{config_opt}章および表\ref{tbl:realtype}参照)に対応した実数型になります.

この文書内で説明のあるFortran90インターフェイスのreal型については,
configure時オプションで指定したREAL\_TYPEと同じ型を使用してください.\\

\begin{spacing}{0.8}
\begin{itemize}
\item[・] {{\tt --with-real=float}指定もしくは未指定の場合}\\
real型は,real*4もしくはデフォルトkind=4としたrealを使用する.\\
\item[・] {{\tt --with-real=double}指定の場合}\\
real型は,real*8もしくはデフォルトkind=8としたrealを使用する.\\
\end{itemize}
\end{spacing}


\subsection{MPIリクエスト番号}
\label{reqNoF}
CPMライブラリのFortran90インターフェイスが提供するMPI通信関数は,ライブラリ内部で
C++コードからMPI関数を呼び出しています.

Fortran90では,C++で扱うMPI\_Request型を扱うことができないため,Fortran90インターフェイスの
非同期通信関数で用いるMPIリクエストは,CPMライブラリ内部で番号管理しています.


\subsection{サンプルプログラム}

Examples/f90ディレクトリに,Fortran90ユーザープログラムでの
CPMライブラリ使用例のサンプルソースコードが収められています.

\begin{itemize}
\item[・] {\tt mconvp\_CPM.f90} \\
入力ファイルの読み込み,全ランクへの展開,領域分割実行を行い,poisson方程式を計算する
サンプルです.\\
このサンプルコードはmake時に一緒にコンパイル,リンクされ,{\tt mconvp\_CPM}という
実行ファイルが作成されます.{\tt data}ディレクトリに実行サンプルが収められています.
\end{itemize}


\subsection{cpm\_fparam.fiのインクルード}

CPMライブラリのFortran90インターフェイスで使用する各種定義は,CPMライブラリが
提供するヘッダファイルcpm\_fparam.fiで定義されています.
CPMライブラリのFortran90インターフェイスを使う場合は,このcpm\_fparam.fiファイルを
インクルードします.

cpm\_fparam.fiは,configureスクリプト実行時の設定{\tt prefix}
配下の{\tt \$\{prefix\}}/includeに{\tt make install}時にインストールされます.\\


\subsection{parameter定義}

CPMライブラリのFortran90インターフェイスで使用される定数は,
cpm\_fparam.fiに定義されています.

\begin{itemize}
\item[・] {戻り値}

Fortran90インターフェイス関数では,引数に必ず処理の戻り値を設定します.
処理が正常終了した場合,この戻り値には0がセットされます.
インターフェイス関数内でエラーが発生した場合,この戻り値にはcpm\_ErrorCode列挙型
(\ref{tbl:ErrorCode1},\ref{tbl:ErrorCode2}を参照)の値が整数値として
セットされます.\\
cpm\_fparam.fiでは,正常終了の場合の戻り値0をCPM\_SUCCESS定数として定義しています.
(表\ref{tbl:retval})\\

\begin{table}[htb]
\begin{center}
\caption{戻り値定数}
\label{tbl:retval}
\begin{tabular}{|c|c|l|}
\hline 
定数名 & 値 & 意味\\
\hline
CPM\_SUCCESS & 0 & 正常終了 \\
\hline   
\end{tabular}
\end{center}
\end{table}


\item[・] {デフォルトのプロセスグループ番号}

CPMライブラリは,プロセスグループの機能を提供しており,並列プロセス内に
存在する複数の計算空間をプロセスグループとして定義しています.
プロセスグループはCPMライブラリ内部で番号で管理されており,デフォルトの
グループ番号は0とされています.\\
cpm\_fparam.fiでは,このデフォルトのプロセスグループ番号を,CPM\_DEFAULT\_GROUP定数
として定義しています.
(表\ref{tbl:defGrp})\\

\begin{table}[htb]
\begin{center}
\caption{プロセスグループ番号定数}
\label{tbl:defGrp}
\begin{tabular}{|c|c|l|}
\hline 
定数名 & 値 & 意味\\
\hline
CPM\_DEFAULT\_GROUP & 0 & デフォルトのプロセスグループ番号 \\
\hline   
\end{tabular}
\end{center}
\end{table}


\item[・] {面フラグ}

CPMライブラリから取得した隣接領域番号配列等の6wordの配列を参照する際の,
配列インデクスの定義です.\\
配列の宣言方法により,2種類のインデクス定義を使い分ける必要があります.
(表\ref{tbl:FaceFlagF0},\ref{tbl:FaceFlagF1})

\begin{table}[htb]
\begin{center}
\caption{面フラグ定数(0スタート)}
\label{tbl:FaceFlagF0}
\begin{tabular}{|c|c|l|}
\hline 
定数名 & 値 & 意味\\
\hline
X\_MINUS & 0 & -X面\\
Y\_MINUS & 1 & -Y面\\
Z\_MINUS & 2 & -Z面\\
X\_PLUS  & 3 & +X面\\
Y\_PLUS  & 4 & +Y面\\
Z\_PLUS  & 5 & +Z面\\
\hline
\end{tabular}
\end{center}
\end{table}

\begin{table}[htb]
\begin{center}
\caption{面フラグ定数(1スタート)}
\label{tbl:FaceFlagF1}
\begin{tabular}{|c|c|l|}
\hline 
定数名 & 値 & 意味\\
\hline
X\_MINUS1 & 1 & -X面\\
Y\_MINUS1 & 2 & -Y面\\
Z\_MINUS1 & 3 & -Z面\\
X\_PLUS1  & 4 & +X面\\
Y\_PLUS1  & 5 & +Y面\\
Z\_PLUS1  & 6 & +Z面\\
\hline
\end{tabular}
\end{center}
\end{table}


\item[・] {軸方向フラグ}

CPMライブラリから取得したVOXEL数配列等の3wordの配列を参照する際の,配列インデクスの定義です.
また,周期境界通信関数の周期境界方向を指定するフラグとしても使われます.
(表\ref{tbl:DirFlagF})\\

\begin{table}[htb]
\begin{center}
\caption{軸方向フラグ定数}
\label{tbl:DirFlagF}
\begin{tabular}{|c|c|l|}
\hline 
定数名 & 値 & 意味\\
\hline
X\_DIR & 0 & X方向\\
Y\_DIR & 1 & Y方向\\
Z\_DIR & 2 & Z方向\\
\hline
\end{tabular}
\end{center}
\end{table}


\item[・] {正負方向フラグ}

周期境界通信関数の周期境界方向を指定するフラグとして使われます.
(表\ref{tbl:PMFlagF})\\

\begin{table}[htb]
\begin{center}
\caption{正負方向フラグ定数}
\label{tbl:PMFlagF}
\begin{tabular}{|c|c|l|}
\hline 
定数名 & 値 & 意味\\
\hline
PLUS2MINUS & 0 & +側から-側\\
MINUS2PLUS & 1 & -側から+側\\
BOTH       & 2 & 双方向\\
\hline
\end{tabular}
\end{center}
\end{table}


\item[・] {Fortranデータ型}

CPMライブラリが提供するMPI通信関係のFortran90インターフェイスでは,
通信データのデータ型を指定する必要があります.\\
データ型はMPI\_Datatypeに対応した独自定数として定義されています.
(表\ref{tbl:dataTypeF})\\

\begin{table}[htb]
\begin{center}
\caption{Fortanデータ型定数}
\label{tbl:dataTypeF}
\begin{tabular}{|c|c|l|}
\hline 
定数名 & 値 & 意味\\
\hline
CPM\_INT     &  6 & MPI\_INTEGERに対応\\
CPM\_INTEGER &  6 & MPI\_INTEGERに対応\\
CPM\_REAL    & 52 & REAL\_TYPEに対応\\
CPM\_FLOAT   & 10 & MPI\_REAL4に対応\\
CPM\_REAL4   & 10 & MPI\_REAL4に対応\\
CPM\_DOUBLE  & 11 & MPI\_REAL8に対応\\
CPM\_REAL8   & 11 & MPI\_REAL8に対応\\
\hline
\end{tabular}
\end{center}
\end{table}


\item[・] {Fortranオペレータタイプ}

CPMライブラリが提供するreduce通信関係のFortran90インターフェイスでは,
reduceのオペレータタイプを指定する必要があります.\\
オペレータタイプはMPI\_Opに対応した独自定数として定義されています.
(表\ref{tbl:opF})\\

\begin{table}[htb]
\begin{center}
\caption{Fortranオペレータタイプ}
\label{tbl:opF}
\begin{tabular}{|c|c|l|}
\hline 
定数名 & 値 & 意味\\
\hline
CPM\_MAX  & 100 & MPI\_MAXに対応 \\
CPM\_MIN  & 101 & MPI\_MINに対応 \\
CPM\_SUM  & 102 & MPI\_SUMに対応 \\
CPM\_PROD & 103 & MPI\_PRODに対応 \\
CPM\_LAND & 104 & MPI\_LANDに対応 \\
CPM\_BAND & 105 & MPI\_BANDに対応 \\
CPM\_LOR  & 106 & MPI\_LORに対応 \\
CPM\_BOR  & 107 & MPI\_BORに対応 \\
CPM\_LXOR & 108 & MPI\_LXORに対応 \\
CPM\_BXOR & 109 & MPI\_BXORに対応 \\
\hline
\end{tabular}
\end{center}
\end{table}
\end{itemize}


\clearpage


\subsection{CPMライブラリ初期化処理}
\label{CPMinitF}

CPMライブラリを利用する上で,プログラムの先頭で1回だけ初期化メソッドを
呼び出す必要があります.

\begin{spacing}{0.8}
\begin{itembox}[l]{CPMライブラリ初期化処理}
{\tt
\begin{verbatim}
subroutine cpm_Initialize( ierr )
\end{verbatim}
}
CPMライブラリ初期化処理を行う.\\
この関数をコールする前に,ユーザーのFortran90プログラム内で
MPI\_INITがコールされている必要がある.
\begin{description}
\item[integer {\tt ierr[{\it output}]}] エラーコード(表\ref{tbl:ErrorCode1},\ref{tbl:ErrorCode2}を参照)
\end{description}
\end{itembox}\\
\end{spacing}


\subsection{領域分割処理}
\label{voxelInitF}
領域分割処理を行うメソッドは,引数の違いによる複数のメソッドが
用意されており,次のように定義されています\\

\begin{spacing}{0.8}
\begin{itembox}[l]{領域分割処理}
{\tt
\begin{verbatim}
subroutine cpm_VoxelInit( div, vox, origin, pitch, obcid, maxVC, maxN
                        , procGrpNo, ierr )
\end{verbatim}
}
領域分割処理を行う.\\
引数で指定したX,Y,Z方向の領域分割数,VOXEL数,空間原点座標,VOXELピッチ,
外部境界の境界条件IDを用いた領域分割を行う.\\
領域分割数と,procGrpNoで指定されたランク数が一致している必要があり,
このメソッドを用いて領域分割を行った場合,全てのサブドメインが活性サブドメインとなる.
\begin{description}
\item[integer {\tt div[{\it input}]}] X,Y,Z方向の領域分割数(3wordの配列)
\item[integer {\tt vox[{\it input}]}] X,Y,Z方向の全体VOXEL数(3wordの配列)
\item[real    {\tt origin[{\it input}]}] X,Y,Z方向の全体空間原点座標(3wordの配列)
\item[real    {\tt pitch[{\it input}]}] X,Y,Z方向のVOXELピッチ(3wordの配列)
\item[integer {\tt obcid[{\it input}]}] 外部境界名(6面)の境界条件ID配列(6wordの配列)
\item[integer {\tt maxVC[{\it input}]}] 袖通信バッファ確保用の最大袖層数
\item[integer {\tt maxN[{\it input}]}] 袖通信バッファ確保用の最大成分数
\item[integer {\tt procGrpNo[{\it input}]}] 領域分割を行うプロセスグループの番号
\item[integer {\tt ierr[{\it output}]}] エラーコード(表\ref{tbl:ErrorCode1},\ref{tbl:ErrorCode2}を参照)
\end{description}
\end{itembox}\\
\end{spacing}

\begin{spacing}{0.8}
\begin{itembox}[l]{領域分割処理}
{\tt
\begin{verbatim}
subroutine cpm_VoxelInit_nodiv( vox, origin, pitch, obcid, maxVC, maxN
                              , procGrpNo, ierr )
\end{verbatim}
}
領域分割処理を行う.\\
指定したプロセスグループのランク数で,自動的に領域分割数を決定し,
引数で指定したX,Y,Z方向のVOXEL数,空間原点座標,VOXELピッチ,外部境界の
境界条件IDを用いた領域分割を行う.\\
このメソッドを用いて領域分割を行った場合,全てのサブドメインが活性サブドメインとなる.\\
領域分割数は,隣接領域間の袖通信点数の総数が最小値になるような最適な分割数となる.
\begin{description}
\item[integer {\tt vox[{\it input}]}] X,Y,Z方向の全体VOXEL数(3wordの配列)
\item[real    {\tt origin[{\it input}]}] X,Y,Z方向の全体空間原点座標(3wordの配列)
\item[real    {\tt pitch[{\it input}]}] X,Y,Z方向のVOXELピッチ(3wordの配列)
\item[integer {\tt obcid[{\it input}]}] 外部境界名(6面)の境界条件ID配列(6wordの配列)
\item[integer {\tt maxVC[{\it input}]}] 袖通信バッファ確保用の最大袖層数
\item[integer {\tt maxN[{\it input}]}] 袖通信バッファ確保用の最大成分数
\item[integer {\tt procGrpNo[{\it input}]}] 領域分割を行うプロセスグループの番号
\item[integer {\tt ierr[{\it output}]}] エラーコード(表\ref{tbl:ErrorCode1},\ref{tbl:ErrorCode2}を参照)
\end{description}
\end{itembox}\\
\end{spacing}


\clearpage


\subsection{並列情報の取得}
\label{paraInfoF}
並列関連の各種情報の取得関数は,次のように定義されています.\\

\begin{spacing}{0.8}
\begin{itembox}[l]{並列実行であるかチェックする}
{\tt
\begin{verbatim}
subroutine cpm_IsParallel( ipara, ierr )
\end{verbatim}
}
並列実行であるかチェックする.\\
並列実行時はiparaに1がセットされ,逐次実行時はiparaに1以外がセットされる.\\
mpirun等で実行していても,並列数が1のときは逐次実行と判断される.
\begin{description}
\item[integer {\tt ipara[{\it output}]}] 1:並列実行,1以外:逐次実行
\item[integer {\tt ierr[{\it output}]}] エラーコード(表\ref{tbl:ErrorCode1},\ref{tbl:ErrorCode2}を参照)
\end{description}
\end{itembox}\\
\end{spacing}

\begin{spacing}{0.8}
\begin{itembox}[l]{ランク数の取得}
{\tt
\begin{verbatim}
subroutine cpm_GetNumRank( nrank, procGrpNo, ierr )
\end{verbatim}
}
指定したプロセスグループのランク数を取得する.
\begin{description}
\item[integer {\tt nrank[{\it output}]}] ランク数
\item[integer {\tt procGrpNo[{\it input}]}] プロセスグループ番号
\item[integer {\tt ierr[{\it output}]}] エラーコード(表\ref{tbl:ErrorCode1},\ref{tbl:ErrorCode2}を参照)
\end{description}
\end{itembox}\\
\end{spacing}

\begin{spacing}{0.8}
\begin{itembox}[l]{ランク番号の取得}
{\tt
\begin{verbatim}
subroutine cpm_GetMyRankID( idm procGrpNo, ierr )
\end{verbatim}
}
指定したプロセスグループ内の自分自身のランク番号を取得する.
\begin{description}
\item[integer {\tt id[{\it output}]}] ランク番号
\item[integer {\tt procGrpNo[{\it input}]}] プロセスグループ番号
\item[integer {\tt ierr[{\it output}]}] エラーコード(表\ref{tbl:ErrorCode1},\ref{tbl:ErrorCode2}を参照)
\end{description}
\end{itembox}\\
\end{spacing}

\begin{spacing}{0.8}
\begin{itembox}[l]{領域分割数の取得}
{\tt
\begin{verbatim}
subroutine cpm_GetDivNum( div, procGrpNo, ierr )
\end{verbatim}
}
指定したプロセスグループの領域分割数を取得する.
\begin{description}
\item[integer {\tt div[{\it output}]}] 領域分割数(3wordの配列)
\item[integer {\tt procGrpNo[{\it input}]}] プロセスグループ番号
\item[integer {\tt ierr[{\it output}]}] エラーコード(表\ref{tbl:ErrorCode1},\ref{tbl:ErrorCode2}を参照)
\end{description}
\end{itembox}\\
\end{spacing}

\begin{spacing}{0.8}
\begin{itembox}[l]{自ランクの領域分割位置を取得}
{\tt
\begin{verbatim}
subroutine cpm_GetDivPos( pos, procGrpNo, ierr )
\end{verbatim}
}
指定したプロセスグループ内での領域分割位置を取得する.
\begin{description}
\item[integer {\tt pos[{\it output}]}] 領域分割位置(3wordの配列)
\item[integer {\tt procGrpNo[{\it input}]}] プロセスグループ番号
\item[integer {\tt ierr[{\it output}]}] エラーコード(表\ref{tbl:ErrorCode1},\ref{tbl:ErrorCode2}を参照)
\end{description}
\end{itembox}\\
\end{spacing}


\clearpage


\subsection{全体空間の領域情報取得}
\label{globalInfoF}
全体空間の領域情報取得関数は,cpm\_ParaManager.h 内で次のように定義されています.\\

\begin{spacing}{0.8}
\begin{itembox}[l]{ピッチの取得}
{\tt
\begin{verbatim}
subroutine cpm_GetPitch( pch, procGrpNo, ierr )
\end{verbatim}
}
指定したプロセスグループ内でのVOXELピッチを取得する.
\begin{description}
\item[real    {\tt pch[{\it output}]}] VOXELピッチ(3wordの配列)
\item[integer {\tt procGrpNo[{\it input}]}] プロセスグループ番号
\item[integer {\tt ierr[{\it output}]}] エラーコード(表\ref{tbl:ErrorCode1},\ref{tbl:ErrorCode2}を参照)
\end{description}
\end{itembox}\\
\end{spacing}

\begin{spacing}{0.8}
\begin{itembox}[l]{全体空間ボクセル数を取得}
{\tt
\begin{verbatim}
subroutine cpm_GetGlobalVoxelSize( wsz, procGrpNo, ierr )
\end{verbatim}
}
指定したプロセスグループ内での全体空間のVOXEL数を取得する.
\begin{description}
\item[integer {\tt wsz[{\it output}]}] VOXEL数(3wordの配列)
\item[integer {\tt procGrpNo[{\it input}]}] プロセスグループ番号
\item[integer {\tt ierr[{\it output}]}] エラーコード(表\ref{tbl:ErrorCode1},\ref{tbl:ErrorCode2}を参照)
\end{description}
\end{itembox}\\
\end{spacing}

\begin{spacing}{0.8}
\begin{itembox}[l]{全体空間の原点座標を取得}
{\tt
\begin{verbatim}
subroutine cpm_GetGlobalOrigin( worg, procGrpNo, ierr )
\end{verbatim}
}
指定したプロセスグループ内での全体空間の原点座標を取得する.
\begin{description}
\item[real    {\tt worg[{\it output}]}] 原点座標(3wordの配列)
\item[integer {\tt procGrpNo[{\it input}]}] プロセスグループ番号
\item[integer {\tt ierr[{\it output}]}] エラーコード(表\ref{tbl:ErrorCode1},\ref{tbl:ErrorCode2}を参照)
\end{description}
\end{itembox}\\
\end{spacing}

\begin{spacing}{0.8}
\begin{itembox}[l]{全体空間の空間サイズを取得}
{\tt
\begin{verbatim}
subroutine cpm_GetGlobalRegion( wrgn, procGrpNo, ierr )
\end{verbatim}
}
指定したプロセスグループ内での全体空間の空間サイズを取得する.
\begin{description}
\item[real    {\tt wrgn[{\it output}]}] 空間サイズ(3wordの配列)
\item[integer {\tt procGrpNo[{\it input}]}] プロセスグループ番号
\item[integer {\tt ierr[{\it output}]}] エラーコード(表\ref{tbl:ErrorCode1},\ref{tbl:ErrorCode2}を参照)
\end{description}
\end{itembox}\\
\end{spacing}


\clearpage


\subsection{ローカル空間の領域情報取得}
\label{localInfoF}
ローカル空間の領域情報取得関数は,次のように定義されています.\\

\begin{spacing}{0.8}
\begin{itembox}[l]{自ランクのボクセル数を取得}
{\tt
\begin{verbatim}
subroutine cpm_GetLocalVoxelSize( lsz, procGrpNo, ierr )
\end{verbatim}
}
指定したプロセスグループ内での自ランクのVOXEL数を取得する.
\begin{description}
\item[integer {\tt lsz[{\it output}]}] VOXEL数(3wordの配列)
\item[integer {\tt procGrpNo[{\it input}]}] プロセスグループ番号
\item[integer {\tt ierr[{\it output}]}] エラーコード(表\ref{tbl:ErrorCode1},\ref{tbl:ErrorCode2}を参照)
\end{description}
\end{itembox}\\
\end{spacing}

\begin{spacing}{0.8}
\begin{itembox}[l]{自ランクの原点座標を取得}
{\tt
\begin{verbatim}
subroutine cpm_GetLocalOrigin( lorg, procGrpNo, ierr )
\end{verbatim}
}
指定したプロセスグループ内での自ランクの原点座標を取得する.
\begin{description}
\item[real    {\tt lorg[{\it output}]}] 原点座標(3wordの配列)
\item[integer {\tt procGrpNo[{\it input}]}] プロセスグループ番号
\item[integer {\tt ierr[{\it output}]}] エラーコード(表\ref{tbl:ErrorCode1},\ref{tbl:ErrorCode2}を参照)
\end{description}
\end{itembox}\\
\end{spacing}

\begin{spacing}{0.8}
\begin{itembox}[l]{自ランクの空間サイズを取得}
{\tt
\begin{verbatim}
subroutine cpm_GetLocalRegion( lrgn, procGrpNo, ierr )
\end{verbatim}
}
指定したプロセスグループ内での自ランクの空間サイズを取得する.
\begin{description}
\item[real    {\tt lrgn[{\it output}]}] 空間サイズ(3wordの配列)
\item[integer {\tt procGrpNo[{\it input}]}] プロセスグループ番号
\item[integer {\tt ierr[{\it output}]}] エラーコード(表\ref{tbl:ErrorCode1},\ref{tbl:ErrorCode2}を参照)
\end{description}
\end{itembox}\\
\end{spacing}

\begin{spacing}{0.8}
\begin{itembox}[l]{自ランクのBCIDを取得}
{\tt
\begin{verbatim}
subroutine cpm_GetBCID( bcid, procGrpNo, ierr )
\end{verbatim}
}
指定したプロセスグループ内での自ランクのBCIDを取得する.
\begin{description}
\item[integer {\tt bcid[{\it output}]}] BCID(6wordの配列)
\item[integer {\tt procGrpNo[{\it input}]}] プロセスグループ番号
\item[integer {\tt ierr[{\it output}]}] エラーコード(表\ref{tbl:ErrorCode1},\ref{tbl:ErrorCode2}を参照)
\end{description}
\end{itembox}\\
\end{spacing}

\begin{spacing}{0.8}
\begin{itembox}[l]{自ランクの始点VOXELの全体空間でのインデクスを取得}
{\tt
\begin{verbatim}
subroutine cpm_GetVoxelHeadIndex( idx, procGrpNo, ierr )
\end{verbatim}
}
指定したプロセスグループ内での,自ランクの始点VOXELの全体空間でのインデクスを取得する.\\
全体空間における始点VOXELのインデクスを0としたインデクスが取得される.
\begin{description}
\item[integer {\tt idx[{\it output}]}] 始点VOXELインデクス(3wordの配列)
\item[integer {\tt procGrpNo[{\it input}]}] プロセスグループ番号
\item[integer {\tt ierr[{\it output}]}] エラーコード(表\ref{tbl:ErrorCode1},\ref{tbl:ErrorCode2}を参照)
\end{description}
\end{itembox}\\
\end{spacing}

\begin{spacing}{0.8}
\begin{itembox}[l]{自ランクの終点VOXELの全体空間でのインデクスを取得}
{\tt
\begin{verbatim}
subroutine cpm_GetVoxelTailIndex( idx, procGrpNo, ierr )
\end{verbatim}
}
指定したプロセスグループ内での,自ランクの終点VOXELの全体空間でのインデクスを取得する.\\
全体空間における始点VOXELのインデクスを0としたインデクスが取得される.
\begin{description}
\item[integer {\tt idx[{\it output}]}] 終点VOXELインデクス(3wordの配列)
\item[integer {\tt procGrpNo[{\it input}]}] プロセスグループ番号
\item[integer {\tt ierr[{\it output}]}] エラーコード(表\ref{tbl:ErrorCode1},\ref{tbl:ErrorCode2}を参照)
\end{description}
\end{itembox}\\
\end{spacing}

\begin{spacing}{0.8}
\begin{itembox}[l]{自ランクの隣接ランク番号を取得}
{\tt
\begin{verbatim}
subroutine cpm_GetNeighborRankID( nID, procGrpNo, ierr )
\end{verbatim}
}
指定したプロセスグループ内での自ランクの隣接ランク番号を取得する.\\
隣接領域が存在しない面方向には,NULLのランクがセットされている.
\begin{description}
\item[integer {\tt nID[{\it output}]}] 隣接ランク番号(6wordの配列)
\item[integer {\tt procGrpNo[{\it input}]}] プロセスグループ番号
\item[integer {\tt ierr[{\it output}]}] エラーコード(表\ref{tbl:ErrorCode1},\ref{tbl:ErrorCode2}を参照)
\end{description}
\end{itembox}\\
\end{spacing}

\begin{spacing}{0.8}
\begin{itembox}[l]{自ランクの周期境界位置の隣接ランク番号を取得}
{\tt
\begin{verbatim}
subroutine cpm_GetPeriodicRankID( nID, procGrpNo, ierr )
\end{verbatim}
}
指定したプロセスグループ内での自ランクの周期境界の隣接ランク番号を取得する.\\
内部境界および周期境界位置に隣接領域が存在しない面方向には,NULLのランクが
セットされている.
\begin{description}
\item[integer {\tt nID[{\it output}]}] 周期境界位置の隣接ランク番号(6wordの配列)
\item[integer {\tt procGrpNo[{\it input}]}] プロセスグループ番号
\item[integer {\tt ierr[{\it output}]}] エラーコード(表\ref{tbl:ErrorCode1},\ref{tbl:ErrorCode2}を参照)
\end{description}
\end{itembox}\\
\end{spacing}


\clearpage


\subsection{MPI通信関数}
\label{mpifuncF}
CPMライブラリでは,プロセスグループ内での並列通信処理メソッドを提供しています.

プロセスグループ内での並列通信処理メソッドは,MPI関数をラップする形で実装されており,
MPI関数とほぼ同じインターフェイスで利用することができます.

プロセスグループ内での並列通信処理メソッドは,次のように定義されています.\\

\begin{spacing}{0.8}
\begin{itembox}[l]{MPI\_Abortのインターフェイス}
{\tt
\begin{verbatim}
subroutine cpm_Abort( errorcode )
\end{verbatim}
}
MPI\_Abortのインターフェイスメソッド.
\begin{description}
\item[{\tt errorcode[{\it input}]}] MPI\_Abortに渡すエラーコード
\end{description}
\end{itembox}\\
\end{spacing}

\begin{spacing}{0.8}
\begin{itembox}[l]{MPI\_Barrierのインターフェイス}
{\tt
\begin{verbatim}
subroutine cpm_Barrier( procGrpNo, ierr )
\end{verbatim}
}
MPI\_Barrierのインターフェイスメソッド.
\begin{description}
\item[integer {\tt procGrpNo[{\it input}]}] プロセスグループ番号
\item[integer {\tt ierr[{\it output}]}] エラーコード(表\ref{tbl:ErrorCode1},\ref{tbl:ErrorCode2}を参照)
\end{description}
\end{itembox}\\
\end{spacing}

\begin{spacing}{0.8}
\begin{itembox}[l]{MPI\_Waitのインターフェイス}
{\tt
\begin{verbatim}
subroutine cpm_Wait( reqNo, ierr )
\end{verbatim}
}
MPI\_Waitのインターフェイスメソッド.
\begin{description}
\item[integer {\tt reqNo[{\it input}]}] リクエスト番号(\ref{reqNoF}章を参照)
\item[integer {\tt ierr[{\it output}]}] エラーコード(表\ref{tbl:ErrorCode1},\ref{tbl:ErrorCode2}を参照)
\end{description}
\end{itembox}\\
\end{spacing}

\begin{spacing}{0.8}
\begin{itembox}[l]{MPI\_Waitallのインターフェイス}
{\tt
\begin{verbatim}
subroutine cpm_Waitall( count, reqlist, ierr )
\end{verbatim}
}
MPI\_Waitallのインターフェイスメソッド.
\begin{description}
\item[integer {\tt count[{\it input}]}] MPI\_Request数
\item[integer {\tt reqlist[{\it input}]}] リクエスト番号配列(\ref{reqNoF}章を参照)
\item[integer {\tt ierr[{\it output}]}] エラーコード(表\ref{tbl:ErrorCode1},\ref{tbl:ErrorCode2}を参照)
\end{description}
\end{itembox}\\
\end{spacing}

\begin{spacing}{0.8}
\begin{itembox}[l]{MPI\_Bcastのインターフェイス}
{\tt
\begin{verbatim}
subroutine cpm_Bcast( buf, count, datatype, root, procGrpNo, ierr )
\end{verbatim}
}
MPI\_Bcastのインターフェイスメソッド.
\begin{description}
\item[void    {\tt buf[{\it input/output}]}] 送受信バッファ
\item[integer {\tt count[{\it input}]}] 送受信する配列要素数
\item[integer {\tt datatype[{\it input}]}] 送受信バッファのデータタイプ(表\ref{tbl:dataTypeF}を参照)
\item[integer {\tt root[{\it input}]}] 送信元ランク番号(procGrpNo内でのランク番号)
\item[integer {\tt procGrpNo[{\it input}]}] プロセスグループ番号
\item[integer {\tt ierr[{\it output}]}] エラーコード(表\ref{tbl:ErrorCode1},\ref{tbl:ErrorCode2}を参照)
\end{description}
\end{itembox}\\
\end{spacing}

\begin{spacing}{0.8}
\begin{itembox}[l]{MPI\_Sendのインターフェイス}
{\tt
\begin{verbatim}
subroutine cpm_Send( buf, count, datatype, dest, procGrpNo, ierr )
\end{verbatim}
}
MPI\_Sendのインターフェイスメソッド.
\begin{description}
\item[void    {\tt buf[{\it input}]}] 送信バッファ
\item[integer {\tt count[{\it input}]}] 送信する配列要素数
\item[integer {\tt datatype[{\it input}]}] 送信バッファのデータタイプ(表\ref{tbl:dataTypeF}を参照)
\item[integer {\tt dest[{\it input}]}] 送信先ランク番号(procGrpNo内でのランク番号)
\item[integer {\tt procGrpNo[{\it input}]}] プロセスグループ番号
\item[integer {\tt ierr[{\it output}]}] エラーコード(表\ref{tbl:ErrorCode1},\ref{tbl:ErrorCode2}を参照)
\end{description}
\end{itembox}\\
\end{spacing}

\begin{spacing}{0.8}
\begin{itembox}[l]{MPI\_Recvのインターフェイス}
{\tt
\begin{verbatim}
subroutine cpm_Recv( buf, count, datatype, source, procGrpNo, ierr )
\end{verbatim}
}
MPI\_Recvのインターフェイスメソッド.
\begin{description}
\item[void    {\tt buf[{\it output}]}] 受信バッファ
\item[integer {\tt count[{\it input}]}] 受信する配列要素数
\item[integer {\tt datatype[{\it input}]}] 受信バッファのデータタイプ(表\ref{tbl:dataTypeF}を参照)
\item[integer {\tt source[{\it input}]}] 送信元ランク番号(procGrpNo内でのランク番号)
\item[integer {\tt procGrpNo[{\it input}]}] プロセスグループ番号
\item[integer {\tt ierr[{\it output}]}] エラーコード(表\ref{tbl:ErrorCode1},\ref{tbl:ErrorCode2}を参照)
\end{description}
\end{itembox}\\
\end{spacing}

\begin{spacing}{0.8}
\begin{itembox}[l]{MPI\_Isendのインターフェイス}
{\tt
\begin{verbatim}
subroutine cpm_Isend( buf, count, datatype, dest, procGrpNo, reqNo, ierr )
\end{verbatim}
}
MPI\_Isendのインターフェイスメソッド.
\begin{description}
\item[void    {\tt buf[{\it input}]}] 送信バッファ
\item[integer {\tt count[{\it input}]}] 送信する配列要素数
\item[integer {\tt datatype[{\it input}]}] 送信バッファのデータタイプ(表\ref{tbl:dataTypeF}を参照)
\item[integer {\tt dest[{\it input}]}] 送信先ランク番号(procGrpNo内でのランク番号)
\item[integer {\tt procGrpNo[{\it input}]}] プロセスグループ番号
\item[integer {\tt reqNo[{\it output}]}] リクエスト番号(\ref{reqNoF}章を参照)
\item[integer {\tt ierr[{\it output}]}] エラーコード(表\ref{tbl:ErrorCode1},\ref{tbl:ErrorCode2}を参照)
\end{description}
\end{itembox}\\
\end{spacing}

\begin{spacing}{0.8}
\begin{itembox}[l]{MPI\_Irecvのインターフェイス}
{\tt
\begin{verbatim}
subroutine cpm_Irecv( buf, count, datatype, source, procGrpNo, reqNo
                    , ierr )
\end{verbatim}
}
MPI\_Irecvのインターフェイスメソッド.
\begin{description}
\item[void    {\tt buf[{\it output}]}] 受信バッファ
\item[integer {\tt count[{\it input}]}] 受信する配列要素数
\item[integer {\tt datatype[{\it input}]}] 受信バッファのデータタイプ(表\ref{tbl:dataTypeF}を参照)
\item[integer {\tt source[{\it input}]}] 送信元ランク番号(procGrpNo内でのランク番号)
\item[integer {\tt procGrpNo[{\it input}]}] プロセスグループ番号
\item[integer {\tt reqNo[{\it output}]}] リクエスト番号(\ref{reqNoF}章を参照)
\item[integer {\tt ierr[{\it output}]}] エラーコード(表\ref{tbl:ErrorCode1},\ref{tbl:ErrorCode2}を参照)
\end{description}
\end{itembox}\\
\end{spacing}

\begin{spacing}{0.8}
\begin{itembox}[l]{MPI\_Allreduceのインターフェイス}
{\tt
\begin{verbatim}
subroutine cpm_Allreduce( sendbuf, recvbuf, count, datatype, op, procGrpNo
                        , ierr )
\end{verbatim}
}
MPI\_Allreduceのインターフェイスメソッド.
\begin{description}
\item[void    {\tt sendbuf[{\it input}]}] 送信バッファ
\item[void    {\tt recvbuf[{\it output}]}] 受信バッファ
\item[integer {\tt count[{\it input}]}] 送受信する配列要素数
\item[integer {\tt datatype[{\it input}]}] 送受信バッファのデータタイプ(表\ref{tbl:dataTypeF}を参照)
\item[integer {\tt op[{\it input}]}] オペレータ(表\ref{tbl:opF}を参照)
\item[integer {\tt procGrpNo[{\it input}]}] プロセスグループ番号
\item[integer {\tt ierr[{\it output}]}] エラーコード(表\ref{tbl:ErrorCode1},\ref{tbl:ErrorCode2}を参照)
\end{description}
\end{itembox}\\
\end{spacing}

\begin{spacing}{0.8}
\begin{itembox}[l]{MPI\_Gatherのインターフェイス}
{\tt
\begin{verbatim}
subroutine cpm_Gather( sendbuf, sendcnt, sendtype
                     , recvbuf, recvcnt, recvtype
                     , root, procGrpNo, ierr )
\end{verbatim}
}
MPI\_Gatherのインターフェイスメソッド.
\begin{description}
\item[void    {\tt sendbuf[{\it input}]}] 送信バッファ
\item[integer {\tt sendcnt[{\it input}]}] 送信バッファの配列要素数
\item[integer {\tt sendtype[{\it input}]}] 送信バッファのデータタイプ(表\ref{tbl:dataTypeF}を参照)
\item[void    {\tt recvbuf[{\it output}]}] 受信バッファ
\item[integer {\tt recvcnt[{\it input}]}] 受信バッファの配列要素数
\item[integer {\tt recvtype[{\it input}]}] 受信バッファのデータタイプ(表\ref{tbl:dataTypeF}を参照)
\item[integer {\tt root[{\it input}]}] 受信するランク番号(procGrpNo内でのランク番号)
\item[integer {\tt procGrpNo[{\it input}]}] プロセスグループ番号
\item[integer {\tt ierr[{\it output}]}] エラーコード(表\ref{tbl:ErrorCode1},\ref{tbl:ErrorCode2}を参照)
\end{description}
\end{itembox}\\
\end{spacing}

\begin{spacing}{0.8}
\begin{itembox}[l]{MPI\_Allgatherのインターフェイス}
{\tt
\begin{verbatim}
subroutine cpm_Allgather( sendbuf, sendcnt, sendtype
                        , recvbuf, recvcnt, recvtype
                        , procGrpNo, ierr )
\end{verbatim}
}
MPI\_Allgatherのインターフェイスメソッド.
\begin{description}
\item[void    {\tt sendbuf[{\it input}]}] 送信バッファ
\item[integer {\tt sendcnt[{\it input}]}] 送信バッファの配列要素数
\item[integer {\tt sendtype[{\it input}]}] 送信バッファのデータタイプ(表\ref{tbl:dataTypeF}を参照)
\item[void    {\tt recvbuf[{\it output}]}] 受信バッファ
\item[integer {\tt recvcnt[{\it input}]}] 受信バッファの配列要素数
\item[integer {\tt recvtype[{\it input}]}] 受信バッファのデータタイプ(表\ref{tbl:dataTypeF}を参照)
\item[integer {\tt procGrpNo[{\it input}]}] プロセスグループ番号
\item[integer {\tt ierr[{\it output}]}] エラーコード(表\ref{tbl:ErrorCode1},\ref{tbl:ErrorCode2}を参照)
\end{description}
\end{itembox}\\
\end{spacing}

\begin{spacing}{0.8}
\begin{itembox}[l]{MPI\_Gathervのインターフェイス}
{\tt
\begin{verbatim}
subroutine cpm_Gatherv( sendbuf, sendcnt, sendtype
                      , recvbuf, recvcnts, displs, recvtype
                      , root, procGrpNo, ierr )
\end{verbatim}
}
MPI\_Gathervのインターフェイスメソッド.
\begin{description}
\item[void    {\tt sendbuf[{\it input}]}] 送信バッファ
\item[integer {\tt sendcnt[{\it input}]}] 送信バッファの配列要素数
\item[integer {\tt sendtype[{\it input}]}] 送信バッファのデータタイプ(表\ref{tbl:dataTypeF}を参照)
\item[void    {\tt recvbuf[{\it output}]}] 受信バッファ
\item[integer {\tt recvcnts[{\it input}]}] 各ランクからの受信データサイズ
\item[integer {\tt displs[{\it input}]}] 各ランクからの受信データ配置位置
\item[integer {\tt recvtype[{\it input}]}] 受信バッファのデータタイプ(表\ref{tbl:dataTypeF}を参照)
\item[integer {\tt root[{\it input}]}] 受信するランク番号(procGrpNo内でのランク番号)
\item[integer {\tt procGrpNo[{\it input}]}] プロセスグループ番号
\item[integer {\tt ierr[{\it output}]}] エラーコード(表\ref{tbl:ErrorCode1},\ref{tbl:ErrorCode2}を参照)
\end{description}
\end{itembox}\\
\end{spacing}

\begin{spacing}{0.8}
\begin{itembox}[l]{MPI\_Allgathervのインターフェイス}
{\tt
\begin{verbatim}
subroutine cpm_Allgatherv( sendbuf, sendcnt, sendtype
                         , recvbuf, recvcnts, displs, recvtype
                         , procGrpNo, ierr )
\end{verbatim}
}
MPI\_Allgathervのインターフェイスメソッド.
\begin{description}
\item[void    {\tt sendbuf[{\it input}]}] 送信バッファ
\item[integer {\tt sendcnt[{\it input}]}] 送信バッファの配列要素数
\item[integer {\tt sendtype[{\it input}]}] 送信バッファのデータタイプ(表\ref{tbl:dataTypeF}を参照)
\item[void    {\tt recvbuf[{\it output}]}] 受信バッファ
\item[integer {\tt recvcnts[{\it input}]}] 各ランクからの受信データサイズ
\item[integer {\tt displs[{\it input}]}] 各ランクからの受信データ配置位置
\item[integer {\tt recvtype[{\it input}]}] 受信バッファのデータタイプ(表\ref{tbl:dataTypeF}を参照)
\item[integer {\tt procGrpNo[{\it input}]}] プロセスグループ番号
\item[integer {\tt ierr[{\it output}]}] エラーコード(表\ref{tbl:ErrorCode1},\ref{tbl:ErrorCode2}を参照)
\end{description}
\end{itembox}\\
\end{spacing}


\clearpage


\subsection{内部境界袖通信メソッド}
\label{bndcommF}
CPMライブラリでは,領域分割空間での隣接領域間(内部境界)の袖領域の通信メソッドを
提供しています.袖領域の通信メソッドは,配列形状毎に用意されています.

袖領域の通信メソッドは,次のように定義されています.\\

\begin{spacing}{0.8}
\begin{itembox}[l]{袖通信バッファのセット}
{\tt
\begin{verbatim}
subroutine cpm_SetBndCommBuffer( maxVC, maxN, procGrpNo, ierr )
\end{verbatim}
}
袖通信で使用する転送データ格納用バッファの確保を行う.\\
VoxelInit(\ref{voxelInit}章を参照)実行時にも実行されるが,プログラムの途中で
バッファサイズを変更したい場合に呼び出す.
\begin{description}
\item[integer {\tt maxVC[{\it input}]}] 袖通信バッファ確保用の最大袖層数
\item[integer {\tt maxN[{\it input}]}] 袖通信バッファ確保用の最大成分数
\item[integer {\tt procGrpNo[{\it input}]}] バッファを確保するプロセスグループの番号
\item[integer {\tt ierr[{\it output}]}] エラーコード(表\ref{tbl:ErrorCode1},\ref{tbl:ErrorCode2}を参照)
\end{description}
\end{itembox}\\
\end{spacing}

\begin{spacing}{0.8}
\begin{itembox}[l]{袖通信(Scalar3D版)}
{\tt
\begin{verbatim}
subroutine cpm_BndCommS3D( array, imax, jmax, kmax, vc, vc_comm, datatype
                         , procGrpNo, ierr )
\end{verbatim}
}
Scalar3D形状の配列の内部境界袖通信を行う.\\
VOXEL数が[imax,jmax,kmax]の領域に対して,配列形状がarray(imax,jmax,kmax)の
Fortran型の配列の袖通信を行う.
\begin{description}
\item[void    {\tt array[{\it input/output}]}] 袖通信をする配列の先頭ポインタ
\item[integer {\tt imax[{\it input}]}] 配列サイズ(I方向)
\item[integer {\tt jmax[{\it input}]}] 配列サイズ(J方向)
\item[integer {\tt kmax[{\it input}]}] 配列サイズ(K方向)
\item[integer {\tt vc[{\it input}]}] 仮想セル数
\item[integer {\tt vc\_comm[{\it input}]}] 袖通信を行う仮想セル数
\item[integer {\tt datatype[{\it input}]}] 袖通信をする配列のデータタイプ(表\ref{tbl:dataTypeF}を参照)
\item[integer {\tt procGrpNo[{\it input}]}] プロセスグループ番号
\item[integer {\tt ierr[{\it output}]}] エラーコード(表\ref{tbl:ErrorCode1},\ref{tbl:ErrorCode2}を参照)
\end{description}
\end{itembox}\\
\end{spacing}

\begin{spacing}{0.8}
\begin{itembox}[l]{袖通信(Vector3D版)}
{\tt
\begin{verbatim}
subroutine cpm_BndCommV3D( array, imax, jmax, kmax, vc, vc_comm, datatype
                         , procGrpNo, ierr )
\end{verbatim}
}
Vector3D形状の配列の内部境界袖通信を行う.\\
VOXEL数が[imax,jmax,kmax]の領域に対して,配列形状がarray(imax,jmax,kmax,3)の
Fortran型の配列の袖通信を行う.(成分数=3)
\begin{description}
\item[void    {\tt array[{\it input/output}]}] 袖通信をする配列の先頭ポインタ
\item[integer {\tt imax[{\it input}]}] 配列サイズ(I方向)
\item[integer {\tt jmax[{\it input}]}] 配列サイズ(J方向)
\item[integer {\tt kmax[{\it input}]}] 配列サイズ(K方向)
\item[integer {\tt vc[{\it input}]}] 仮想セル数
\item[integer {\tt vc\_comm[{\it input}]}] 袖通信を行う仮想セル数
\item[integer {\tt datatype[{\it input}]}] 袖通信をする配列のデータタイプ(表\ref{tbl:dataTypeF}を参照)
\item[integer {\tt procGrpNo[{\it input}]}] プロセスグループ番号
\item[integer {\tt ierr[{\it output}]}] エラーコード(表\ref{tbl:ErrorCode1},\ref{tbl:ErrorCode2}を参照)
\end{description}
\end{itembox}\\
\end{spacing}

\begin{spacing}{0.8}
\begin{itembox}[l]{袖通信(Scalar4D版)}
{\tt
\begin{verbatim}
subroutine cpm_BndCommS4D( array, imax, jmax, kmax, nmax, vc, vc_comm
                         , datatype, procGrpNo, ierr )
\end{verbatim}
}
Scalar4D形状の配列の内部境界袖通信を行う.\\
VOXEL数が[imax,jmax,kmax]の領域に対して,配列形状がarray(imax,jmax,kmax,nmax)の
Fortran型の配列の袖通信を行う.(成分数=nmax)
\begin{description}
\item[void    {\tt array[{\it input/output}]}] 袖通信をする配列の先頭ポインタ
\item[integer {\tt imax[{\it input}]}] 配列サイズ(I方向)
\item[integer {\tt jmax[{\it input}]}] 配列サイズ(J方向)
\item[integer {\tt kmax[{\it input}]}] 配列サイズ(K方向)
\item[integer {\tt nmax[{\it input}]}] 成分数
\item[integer {\tt vc[{\it input}]}] 仮想セル数
\item[integer {\tt vc\_comm[{\it input}]}] 袖通信を行う仮想セル数
\item[integer {\tt datatype[{\it input}]}] 袖通信をする配列のデータタイプ(表\ref{tbl:dataTypeF}を参照)
\item[integer {\tt procGrpNo[{\it input}]}] プロセスグループ番号
\item[integer {\tt ierr[{\it output}]}] エラーコード(表\ref{tbl:ErrorCode1},\ref{tbl:ErrorCode2}を参照)
\end{description}
\end{itembox}\\
\end{spacing}

\begin{spacing}{0.8}
\begin{itembox}[l]{袖通信(Vector3DEx版)}
{\tt
\begin{verbatim}
subroutine cpm_BndCommV3DEx( array, imax, jmax, kmax, nmax, vc, vc_comm
                           , datatype, procGrpNo, ierr )
\end{verbatim}
}
Vector3DEx形状の配列の内部境界袖通信を行う.\\
VOXEL数が[imax,jmax,kmax]の領域に対して,配列形状がarray(3,imax,jmax,kmax)の
Fortran型の配列の袖通信を行う.(成分数=3)
\begin{description}
\item[void    {\tt array[{\it input/output}]}] 袖通信をする配列の先頭ポインタ
\item[integer {\tt imax[{\it input}]}] 配列サイズ(I方向)
\item[integer {\tt jmax[{\it input}]}] 配列サイズ(J方向)
\item[integer {\tt kmax[{\it input}]}] 配列サイズ(K方向)
\item[integer {\tt vc[{\it input}]}] 仮想セル数
\item[integer {\tt vc\_comm[{\it input}]}] 袖通信を行う仮想セル数
\item[integer {\tt datatype[{\it input}]}] 袖通信をする配列のデータタイプ(表\ref{tbl:dataTypeF}を参照)
\item[integer {\tt procGrpNo[{\it input}]}] プロセスグループ番号
\item[integer {\tt ierr[{\it output}]}] エラーコード(表\ref{tbl:ErrorCode1},\ref{tbl:ErrorCode2}を参照)
\end{description}
\end{itembox}\\
\end{spacing}

\begin{spacing}{0.8}
\begin{itembox}[l]{袖通信(Scalar4DEx版)}
{\tt
\begin{verbatim}
subroutine cpm_BndCommS4DEx( array, nmax, imax, jmax, kmax, vc, vc_comm
                           , datatype, procGrpNo, ierr )
\end{verbatim}
}
Scalar4DEx形状の配列の内部境界袖通信を行う.\\
VOXEL数が[imax,jmax,kmax]の領域に対して,配列形状がarray(nmax,imax,jmax,kmax)の
Fortran型の配列の袖通信を行う.(成分数=nmax)
\begin{description}
\item[void    {\tt array[{\it input/output}]}] 袖通信をする配列の先頭ポインタ
\item[integer {\tt nmax[{\it input}]}] 成分数
\item[integer {\tt imax[{\it input}]}] 配列サイズ(I方向)
\item[integer {\tt jmax[{\it input}]}] 配列サイズ(J方向)
\item[integer {\tt kmax[{\it input}]}] 配列サイズ(K方向)
\item[integer {\tt vc[{\it input}]}] 仮想セル数
\item[integer {\tt vc\_comm[{\it input}]}] 袖通信を行う仮想セル数
\item[integer {\tt datatype[{\it input}]}] 袖通信をする配列のデータタイプ(表\ref{tbl:dataTypeF}を参照)
\item[integer {\tt procGrpNo[{\it input}]}] プロセスグループ番号
\item[integer {\tt ierr[{\it output}]}] エラーコード(表\ref{tbl:ErrorCode1},\ref{tbl:ErrorCode2}を参照)
\end{description}
\end{itembox}\\
\end{spacing}


\clearpage


\subsection{内部境界袖通信メソッド(非同期版)}
\label{HbndcommF}
非同期版の内部境界袖通信メソッドは,送信データのパックと非同期送受信処理をするメソッドと,
通信完了の待機と受信データの展開をするメソッドに分かれており,非同期通信中に他の
計算処理を実行することが可能です.

非同期版の袖領域通信メソッドは,次のように定義されています.\\

\begin{spacing}{0.8}
\begin{itembox}[l]{非同期袖通信(Scalar3D版)}
{\tt
\begin{verbatim}
subroutine cpm_BndCommS3D_nowait( array, imax, jmax, kmax, vc, vc_comm
                                , datatype, reqlist, procGrpNo, ierr )
\end{verbatim}
}
Scalar3D形状の配列の内部境界袖通信を行う.\\
VOXEL数が[imax,jmax,kmax]の領域に対して,配列形状がarray(imax,jmax,kmax)の
Fortran型の配列の袖通信を行う.\\
通信完了待ちと受信データの展開は行わない.
\begin{description}
\item[void    {\tt array[{\it input}]}] 袖通信をする配列の先頭ポインタ
\item[integer {\tt imax[{\it input}]}] 配列サイズ(I方向)
\item[integer {\tt jmax[{\it input}]}] 配列サイズ(J方向)
\item[integer {\tt kmax[{\it input}]}] 配列サイズ(K方向)
\item[integer {\tt vc[{\it input}]}] 仮想セル数
\item[integer {\tt vc\_comm[{\it input}]}] 袖通信を行う仮想セル数
\item[integer {\tt datatype[{\it input}]}] 袖通信をする配列のデータタイプ(表\ref{tbl:dataTypeF}を参照)
\item[integer {\tt reqlist[{\it output}]}] リクエスト番号配列(サイズ12,\ref{reqNoF}章を参照)
\item[integer {\tt procGrpNo[{\it input}]}] プロセスグループ番号
\item[integer {\tt ierr[{\it output}]}] エラーコード(表\ref{tbl:ErrorCode1},\ref{tbl:ErrorCode2}を参照)
\end{description}
\end{itembox}\\
\end{spacing}

\begin{spacing}{0.8}
\begin{itembox}[l]{非同期袖通信(Scalar3D版)の待機,展開処理}
{\tt
\begin{verbatim}
subroutine cpm_wait_BndCommS3D( array, imax, jmax, kmax, vc, vc_comm
                              , datatype, reqlist, procGrpNo, ierr )
\end{verbatim}
}
cpm\_BndCommS3D\_nowaitメソッドの通信完了待ちと受信データの展開を行う.
\begin{description}
\item[void    {\tt array[{\it input/output}]}] 袖通信をする配列の先頭ポインタ
\item[integer {\tt imax[{\it input}]}] 配列サイズ(I方向)
\item[integer {\tt jmax[{\it input}]}] 配列サイズ(J方向)
\item[integer {\tt kmax[{\it input}]}] 配列サイズ(K方向)
\item[integer {\tt vc[{\it input}]}] 仮想セル数
\item[integer {\tt vc\_comm[{\it input}]}] 袖通信を行う仮想セル数
\item[integer {\tt datatype[{\it input}]}] 袖通信をする配列のデータタイプ(表\ref{tbl:dataTypeF}を参照)
\item[integer {\tt reqlist[{\it input}]}] リクエスト番号配列(サイズ12,\ref{reqNoF}章を参照)
\item[integer {\tt procGrpNo[{\it input}]}] プロセスグループ番号
\item[integer {\tt ierr[{\it output}]}] エラーコード(表\ref{tbl:ErrorCode1},\ref{tbl:ErrorCode2}を参照)
\end{description}
\end{itembox}\\
\end{spacing}

\begin{spacing}{0.8}
\begin{itembox}[l]{非同期袖通信(Vector3D版)}
{\tt
\begin{verbatim}
subroutine cpm_BndCommV3D_nowait( array, imax, jmax, kmax, vc, vc_comm
                                , datatype, reqlist, procGrpNo, ierr )
\end{verbatim}
}
Vector3D形状の配列の内部境界袖通信を行う.\\
VOXEL数が[imax,jmax,kmax]の領域に対して,配列形状がarray(imax,jmax,kmax,3)の
Fortran型の配列の袖通信を行う.(成分数=3)\\
通信完了待ちと受信データの展開は行わない.
\begin{description}
\item[void    {\tt array[{\it input}]}] 袖通信をする配列の先頭ポインタ
\item[integer {\tt imax[{\it input}]}] 配列サイズ(I方向)
\item[integer {\tt jmax[{\it input}]}] 配列サイズ(J方向)
\item[integer {\tt kmax[{\it input}]}] 配列サイズ(K方向)
\item[integer {\tt vc[{\it input}]}] 仮想セル数
\item[integer {\tt vc\_comm[{\it input}]}] 袖通信を行う仮想セル数
\item[integer {\tt datatype[{\it input}]}] 袖通信をする配列のデータタイプ(表\ref{tbl:dataTypeF}を参照)
\item[integer {\tt reqlist[{\it output}]}] リクエスト番号配列(サイズ12,\ref{reqNoF}章を参照)
\item[integer {\tt procGrpNo[{\it input}]}] プロセスグループ番号
\item[integer {\tt ierr[{\it output}]}] エラーコード(表\ref{tbl:ErrorCode1},\ref{tbl:ErrorCode2}を参照)
\end{description}
\end{itembox}\\
\end{spacing}

\begin{spacing}{0.8}
\begin{itembox}[l]{非同期袖通信(Vector3D版)の待機,展開処理}
{\tt
\begin{verbatim}
subroutine cpm_wait_BndCommV3D( array, imax, jmax, kmax, vc, vc_comm
                              , datatype, reqlist, procGrpNo, ierr )
\end{verbatim}
}
cpm\_BndCommV3D\_nowaitメソッドの通信完了待ちと受信データの展開を行う.
\begin{description}
\item[void    {\tt array[{\it input/output}]}] 袖通信をする配列の先頭ポインタ
\item[integer {\tt imax[{\it input}]}] 配列サイズ(I方向)
\item[integer {\tt jmax[{\it input}]}] 配列サイズ(J方向)
\item[integer {\tt kmax[{\it input}]}] 配列サイズ(K方向)
\item[integer {\tt vc[{\it input}]}] 仮想セル数
\item[integer {\tt vc\_comm[{\it input}]}] 袖通信を行う仮想セル数
\item[integer {\tt datatype[{\it input}]}] 袖通信をする配列のデータタイプ(表\ref{tbl:dataTypeF}を参照)
\item[integer {\tt reqlist[{\it input}]}] リクエスト番号配列(サイズ12,\ref{reqNoF}章を参照)
\item[integer {\tt procGrpNo[{\it input}]}] プロセスグループ番号
\item[integer {\tt ierr[{\it output}]}] エラーコード(表\ref{tbl:ErrorCode1},\ref{tbl:ErrorCode2}を参照)
\end{description}
\end{itembox}\\
\end{spacing}

\begin{spacing}{0.8}
\begin{itembox}[l]{非同期袖通信(Scalar4D版)}
{\tt
\begin{verbatim}
subroutine cpm_BndCommS4D_nowait( array, imax, jmax, kmax, nmax, vc
                                , vc_comm, datatype, reqlist, procGrpNo
                                , ierr )
\end{verbatim}
}
Scalar4D形状の配列の内部境界袖通信を行う.\\
VOXEL数が[imax,jmax,kmax]の領域に対して,配列形状がarray(imax,jmax,kmax,nmax)の
Fortran型の配列の袖通信を行う.(成分数=nmax)\\
通信完了待ちと受信データの展開は行わない.
\begin{description}
\item[void    {\tt array[{\it input}]}] 袖通信をする配列の先頭ポインタ
\item[integer {\tt imax[{\it input}]}] 配列サイズ(I方向)
\item[integer {\tt jmax[{\it input}]}] 配列サイズ(J方向)
\item[integer {\tt kmax[{\it input}]}] 配列サイズ(K方向)
\item[integer {\tt nmax[{\it input}]}] 成分数
\item[integer {\tt vc[{\it input}]}] 仮想セル数
\item[integer {\tt vc\_comm[{\it input}]}] 袖通信を行う仮想セル数
\item[integer {\tt datatype[{\it input}]}] 袖通信をする配列のデータタイプ(表\ref{tbl:dataTypeF}を参照)
\item[integer {\tt reqlist[{\it output}]}] リクエスト番号配列(サイズ12,\ref{reqNoF}章を参照)
\item[integer {\tt procGrpNo[{\it input}]}] プロセスグループ番号
\item[integer {\tt ierr[{\it output}]}] エラーコード(表\ref{tbl:ErrorCode1},\ref{tbl:ErrorCode2}を参照)
\end{description}
\end{itembox}\\
\end{spacing}

\begin{spacing}{0.8}
\begin{itembox}[l]{非同期袖通信(Scalar4D版)の待機,展開処理}
{\tt
\begin{verbatim}
subroutine cpm_wait_BndCommS4D( array, imax, jmax, kmax, nmax, vc, vc_comm
                              , datatype, reqlist, procGrpNo, ierr )
\end{verbatim}
}
cpm\_BndCommS4D\_nowaitメソッドの通信完了待ちと受信データの展開を行う.
\begin{description}
\item[void    {\tt array[{\it input/output}]}] 袖通信をする配列の先頭ポインタ
\item[integer {\tt imax[{\it input}]}] 配列サイズ(I方向)
\item[integer {\tt jmax[{\it input}]}] 配列サイズ(J方向)
\item[integer {\tt kmax[{\it input}]}] 配列サイズ(K方向)
\item[integer {\tt nmax[{\it input}]}] 成分数
\item[integer {\tt vc[{\it input}]}] 仮想セル数
\item[integer {\tt vc\_comm[{\it input}]}] 袖通信を行う仮想セル数
\item[integer {\tt datatype[{\it input}]}] 袖通信をする配列のデータタイプ(表\ref{tbl:dataTypeF}を参照)
\item[integer {\tt reqlist[{\it input}]}] リクエスト番号配列(サイズ12,\ref{reqNoF}章を参照)
\item[integer {\tt procGrpNo[{\it input}]}] プロセスグループ番号
\item[integer {\tt ierr[{\it output}]}] エラーコード(表\ref{tbl:ErrorCode1},\ref{tbl:ErrorCode2}を参照)
\end{description}
\end{itembox}\\
\end{spacing}

\begin{spacing}{0.8}
\begin{itembox}[l]{非同期袖通信(Vector3DEx版)}
{\tt
\begin{verbatim}
subroutine cpm_BndCommV3DEx_nowait( array, imax, jmax, kmax, vc, vc_comm
                                  , datatype, reqlist, procGrpNo, ierr )
\end{verbatim}
}
Vector3DEx形状の配列の内部境界袖通信を行う.\\
VOXEL数が[imax,jmax,kmax]の領域に対して,配列形状がarray(3,imax,jmax,kmax)の
Fortran型の配列の袖通信を行う.(成分数=3)\\
通信完了待ちと受信データの展開は行わない.
\begin{description}
\item[void    {\tt array[{\it input}]}] 袖通信をする配列の先頭ポインタ
\item[integer {\tt imax[{\it input}]}] 配列サイズ(I方向)
\item[integer {\tt jmax[{\it input}]}] 配列サイズ(J方向)
\item[integer {\tt kmax[{\it input}]}] 配列サイズ(K方向)
\item[integer {\tt vc[{\it input}]}] 仮想セル数
\item[integer {\tt vc\_comm[{\it input}]}] 袖通信を行う仮想セル数
\item[integer {\tt datatype[{\it input}]}] 袖通信をする配列のデータタイプ(表\ref{tbl:dataTypeF}を参照)
\item[integer {\tt reqlist[{\it output}]}] リクエスト番号配列(サイズ12,\ref{reqNoF}章を参照)
\item[integer {\tt procGrpNo[{\it input}]}] プロセスグループ番号
\item[integer {\tt ierr[{\it output}]}] エラーコード(表\ref{tbl:ErrorCode1},\ref{tbl:ErrorCode2}を参照)
\end{description}
\end{itembox}\\
\end{spacing}

\begin{spacing}{0.8}
\begin{itembox}[l]{非同期袖通信(Vector3DEx版)の待機,展開処理}
{\tt
\begin{verbatim}
subroutine cpm_wait_BndCommV3DEx( array, imax, jmax, kmax, vc, vc_comm
                                , datatype, reqlist, procGrpNo, ierr )
\end{verbatim}
}
cpm\_BndCommV3DEx\_nowaitメソッドの通信完了待ちと受信データの展開を行う.
\begin{description}
\item[void    {\tt array[{\it input/output}]}] 袖通信をする配列の先頭ポインタ
\item[integer {\tt imax[{\it input}]}] 配列サイズ(I方向)
\item[integer {\tt jmax[{\it input}]}] 配列サイズ(J方向)
\item[integer {\tt kmax[{\it input}]}] 配列サイズ(K方向)
\item[integer {\tt vc[{\it input}]}] 仮想セル数
\item[integer {\tt vc\_comm[{\it input}]}] 袖通信を行う仮想セル数
\item[integer {\tt datatype[{\it input}]}] 袖通信をする配列のデータタイプ(表\ref{tbl:dataTypeF}を参照)
\item[integer {\tt reqlist[{\it input}]}] リクエスト番号配列(サイズ12,\ref{reqNoF}章を参照)
\item[integer {\tt procGrpNo[{\it input}]}] プロセスグループ番号
\item[integer {\tt ierr[{\it output}]}] エラーコード(表\ref{tbl:ErrorCode1},\ref{tbl:ErrorCode2}を参照)
\end{description}
\end{itembox}\\
\end{spacing}


\begin{spacing}{0.8}
\begin{itembox}[l]{非同期袖通信(Scalar4DEx版)}
{\tt
\begin{verbatim}
subroutine cpm_BndCommS4DEx_nowait( array, nmax, imax, jmax, kmax, vc
                                  , vc_comm, datatype, reqlist, procGrpNo
                                  , ierr )
\end{verbatim}
}
Scalar4DEx形状の配列の内部境界袖通信を行う.\\
VOXEL数が[imax,jmax,kmax]の領域に対して,配列形状がarray(nmax,imax,jmax,kmax)の
Fortran型の配列の袖通信を行う.(成分数=nmax)\\
通信完了待ちと受信データの展開は行わない.
\begin{description}
\item[void    {\tt array[{\it input}]}] 袖通信をする配列の先頭ポインタ
\item[integer {\tt nmax[{\it input}]}] 成分数
\item[integer {\tt imax[{\it input}]}] 配列サイズ(I方向)
\item[integer {\tt jmax[{\it input}]}] 配列サイズ(J方向)
\item[integer {\tt kmax[{\it input}]}] 配列サイズ(K方向)
\item[integer {\tt vc[{\it input}]}] 仮想セル数
\item[integer {\tt vc\_comm[{\it input}]}] 袖通信を行う仮想セル数
\item[integer {\tt datatype[{\it input}]}] 袖通信をする配列のデータタイプ(表\ref{tbl:dataTypeF}を参照)
\item[integer {\tt reqlist[{\it output}]}] リクエスト番号配列(サイズ12,\ref{reqNoF}章を参照)
\item[integer {\tt procGrpNo[{\it input}]}] プロセスグループ番号
\item[integer {\tt ierr[{\it output}]}] エラーコード(表\ref{tbl:ErrorCode1},\ref{tbl:ErrorCode2}を参照)
\end{description}
\end{itembox}\\
\end{spacing}

\begin{spacing}{0.8}
\begin{itembox}[l]{非同期袖通信(Scalar4DEx版)の待機,展開処理}
{\tt
\begin{verbatim}
subroutine cpm_wait_BndCommS4DEx( array, nmax, imax, jmax, kmax, vc
                                , vc_comm, datatype, reqlist, procGrpNo
                                , ierr )
\end{verbatim}
}
cpm\_BndCommS4DEx\_nowaitメソッドの通信完了待ちと受信データの展開を行う.
\begin{description}
\item[void    {\tt array[{\it input/output}]}] 袖通信をする配列の先頭ポインタ
\item[integer {\tt nmax[{\it input}]}] 成分数
\item[integer {\tt imax[{\it input}]}] 配列サイズ(I方向)
\item[integer {\tt jmax[{\it input}]}] 配列サイズ(J方向)
\item[integer {\tt kmax[{\it input}]}] 配列サイズ(K方向)
\item[integer {\tt vc[{\it input}]}] 仮想セル数
\item[integer {\tt vc\_comm[{\it input}]}] 袖通信を行う仮想セル数
\item[integer {\tt datatype[{\it input}]}] 袖通信をする配列のデータタイプ(表\ref{tbl:dataTypeF}を参照)
\item[integer {\tt reqlist[{\it input}]}] リクエスト番号配列(サイズ12,\ref{reqNoF}章を参照)
\item[integer {\tt procGrpNo[{\it input}]}] プロセスグループ番号
\item[integer {\tt ierr[{\it output}]}] エラーコード(表\ref{tbl:ErrorCode1},\ref{tbl:ErrorCode2}を参照)
\end{description}
\end{itembox}\\
\end{spacing}


\clearpage


\subsection{周期境界袖通信メソッド}
\label{pericommF}
CPMライブラリでは,領域分割空間での周期境界(外部境界)の袖領域の通信メソッドを
提供しています.袖領域の通信メソッドは,配列形状毎に用意されています.

周期境界の袖通信メソッドは,次のように定義されています.

\begin{spacing}{0.8}
\begin{itembox}[l]{周期境界袖通信(Scalar3D版)}
{\tt
\begin{verbatim}
subroutine cpm_PeriodicCommS3D( array, imax, jmax, kmax, vc, vc_comm
                              , dir, pm, datatype, procGrpNo, ierr )
\end{verbatim}
}
Scalar3D形状の配列の周期境界袖通信を行う.\\
VOXEL数が[imax,jmax,kmax]の領域に対して,配列形状がarray(imax,jmax,kmax)の
Fortran型の配列の袖通信を行う.
\begin{description}
\item[void    {\tt array[{\it input/output}]}] 周期境界袖通信をする配列の先頭ポインタ
\item[integer {\tt imax[{\it input}]}] 配列サイズ(I方向)
\item[integer {\tt jmax[{\it input}]}] 配列サイズ(J方向)
\item[integer {\tt kmax[{\it input}]}] 配列サイズ(K方向)
\item[integer {\tt vc[{\it input}]}] 仮想セル数
\item[integer {\tt vc\_comm[{\it input}]}] 袖通信を行う仮想セル数
\item[integer {\tt dir[{\it input}]}] 袖通信を行う軸方向フラグ(表\ref{tbl:DirFlagF}を参照)
\item[integer {\tt pm[{\it input}]}] 袖通信を行う正負方向フラグ(表\ref{tbl:PMFlagF}を参照)
\item[integer {\tt datatype[{\it input}]}] 袖通信をする配列のデータタイプ(表\ref{tbl:dataTypeF}を参照)
\item[integer {\tt procGrpNo[{\it input}]}] プロセスグループ番号
\item[integer {\tt ierr[{\it output}]}] エラーコード(表\ref{tbl:ErrorCode1},\ref{tbl:ErrorCode2}を参照)
\end{description}
\end{itembox}\\
\end{spacing}

\begin{spacing}{0.8}
\begin{itembox}[l]{周期境界袖通信(Vector3D版)}
{\tt
\begin{verbatim}
subroutine cpm_PeriodicCommV3D( array, imax, jmax, kmax, vc, vc_comm
                              , dir, pm, datatype, procGrpNo, ierr )
\end{verbatim}
}
Vector3D形状の配列の周期境界袖通信を行う.\\
VOXEL数が[imax,jmax,kmax]の領域に対して,配列形状がarray(imax,jmax,kmax,3)の
Fortran型の配列の袖通信を行う(成分数=3).
\begin{description}
\item[void    {\tt array[{\it input/output}]}] 周期境界袖通信をする配列の先頭ポインタ
\item[integer {\tt imax[{\it input}]}] 配列サイズ(I方向)
\item[integer {\tt jmax[{\it input}]}] 配列サイズ(J方向)
\item[integer {\tt kmax[{\it input}]}] 配列サイズ(K方向)
\item[integer {\tt vc[{\it input}]}] 仮想セル数
\item[integer {\tt vc\_comm[{\it input}]}] 袖通信を行う仮想セル数
\item[integer {\tt dir[{\it input}]}] 袖通信を行う軸方向フラグ(表\ref{tbl:DirFlagF}を参照)
\item[integer {\tt pm[{\it input}]}] 袖通信を行う正負方向フラグ(表\ref{tbl:PMFlagF}を参照)
\item[integer {\tt datatype[{\it input}]}] 袖通信をする配列のデータタイプ(表\ref{tbl:dataTypeF}を参照)
\item[integer {\tt procGrpNo[{\it input}]}] プロセスグループ番号
\item[integer {\tt ierr[{\it output}]}] エラーコード(表\ref{tbl:ErrorCode1},\ref{tbl:ErrorCode2}を参照)
\end{description}
\end{itembox}\\
\end{spacing}

\begin{spacing}{0.8}
\begin{itembox}[l]{周期境界袖通信(Scalar4D版)}
{\tt
\begin{verbatim}
subroutine cpm_PeriodicCommS4D( array, imax, jmax, kmax, nmax, vc, vc_comm
                              , dir, pm, datatype, procGrpNo, ierr )
\end{verbatim}
}
Scalar4D形状の配列の周期境界袖通信を行う.\\
VOXEL数が[imax,jmax,kmax]の領域に対して,配列形状がarray(imax,jmax,kmax,nmax)の
Fortran型の配列の袖通信を行う(成分数=nmax).
\begin{description}
\item[void    {\tt array[{\it input/output}]}] 周期境界袖通信をする配列の先頭ポインタ
\item[integer {\tt imax[{\it input}]}] 配列サイズ(I方向)
\item[integer {\tt jmax[{\it input}]}] 配列サイズ(J方向)
\item[integer {\tt kmax[{\it input}]}] 配列サイズ(K方向)
\item[integer {\tt nmax[{\it input}]}] 成分数
\item[integer {\tt vc[{\it input}]}] 仮想セル数
\item[integer {\tt vc\_comm[{\it input}]}] 袖通信を行う仮想セル数
\item[integer {\tt dir[{\it input}]}] 袖通信を行う軸方向フラグ(表\ref{tbl:DirFlagF}を参照)
\item[integer {\tt pm[{\it input}]}] 袖通信を行う正負方向フラグ(表\ref{tbl:PMFlagF}を参照)
\item[integer {\tt datatype[{\it input}]}] 袖通信をする配列のデータタイプ(表\ref{tbl:dataTypeF}を参照)
\item[integer {\tt procGrpNo[{\it input}]}] プロセスグループ番号
\item[integer {\tt ierr[{\it output}]}] エラーコード(表\ref{tbl:ErrorCode1},\ref{tbl:ErrorCode2}を参照)
\end{description}
\end{itembox}\\
\end{spacing}

\begin{spacing}{0.8}
\begin{itembox}[l]{周期境界袖通信(Vector3DEx版)}
{\tt
\begin{verbatim}
subroutine cpm_PeriodicCommV3DEx( array, imax, jmax, kmax, vc, vc_comm
                                , dir, pm, datatype, procGrpNo, ierr )
\end{verbatim}
}
Vector3DEx形状の配列の周期境界袖通信を行う.\\
VOXEL数が[imax,jmax,kmax]の領域に対して,配列形状がarray(3,imax,jmax,kmax)の
Fortran型の配列の袖通信を行う(成分数=3).
\begin{description}
\item[void    {\tt array[{\it input/output}]}] 周期境界袖通信をする配列の先頭ポインタ
\item[integer {\tt imax[{\it input}]}] 配列サイズ(I方向)
\item[integer {\tt jmax[{\it input}]}] 配列サイズ(J方向)
\item[integer {\tt kmax[{\it input}]}] 配列サイズ(K方向)
\item[integer {\tt vc[{\it input}]}] 仮想セル数
\item[integer {\tt vc\_comm[{\it input}]}] 袖通信を行う仮想セル数
\item[integer {\tt dir[{\it input}]}] 袖通信を行う軸方向フラグ(表\ref{tbl:DirFlagF}を参照)
\item[integer {\tt pm[{\it input}]}] 袖通信を行う正負方向フラグ(表\ref{tbl:PMFlagF}を参照)
\item[integer {\tt datatype[{\it input}]}] 袖通信をする配列のデータタイプ(表\ref{tbl:dataTypeF}を参照)
\item[integer {\tt procGrpNo[{\it input}]}] プロセスグループ番号
\item[integer {\tt ierr[{\it output}]}] エラーコード(表\ref{tbl:ErrorCode1},\ref{tbl:ErrorCode2}を参照)
\end{description}
\end{itembox}\\
\end{spacing}

\begin{spacing}{0.8}
\begin{itembox}[l]{周期境界袖通信(Scalar4DEx版)}
{\tt
\begin{verbatim}
subroutine cpm_PeriodicCommS4DEx( array, nmax, imax, jmax, kmax, vc
                                , vc_comm, dir, pm, datatype, procGrpNo
                                , ierr )
\end{verbatim}
}
Scalar4DEx形状の配列の周期境界袖通信を行う.\\
VOXEL数が[imax,jmax,kmax]の領域に対して,配列形状がarray(nmax,imax,jmax,kmax)の
Fortran型の配列の袖通信を行う(成分数=nmax).
\begin{description}
\item[void    {\tt array[{\it input/output}]}] 周期境界袖通信をする配列の先頭ポインタ
\item[integer {\tt nmax[{\it input}]}] 成分数
\item[integer {\tt imax[{\it input}]}] 配列サイズ(I方向)
\item[integer {\tt jmax[{\it input}]}] 配列サイズ(J方向)
\item[integer {\tt kmax[{\it input}]}] 配列サイズ(K方向)
\item[integer {\tt vc[{\it input}]}] 仮想セル数
\item[integer {\tt vc\_comm[{\it input}]}] 袖通信を行う仮想セル数
\item[integer {\tt dir[{\it input}]}] 袖通信を行う軸方向フラグ(表\ref{tbl:DirFlagF}を参照)
\item[integer {\tt pm[{\it input}]}] 袖通信を行う正負方向フラグ(表\ref{tbl:PMFlagF}を参照)
\item[integer {\tt datatype[{\it input}]}] 袖通信をする配列のデータタイプ(表\ref{tbl:dataTypeF}を参照)
\item[integer {\tt procGrpNo[{\it input}]}] プロセスグループ番号
\item[integer {\tt ierr[{\it output}]}] エラーコード(表\ref{tbl:ErrorCode1},\ref{tbl:ErrorCode2}を参照)
\end{description}
\end{itembox}\\
\end{spacing}
